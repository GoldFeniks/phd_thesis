\documentclass[../document.tex]{subfiles}

\begin{document}
    \section*{Введение\label{sec::introduction}}
        \inlineparagraph{Актуальность темы исследования}
            В настоящее время в акустике океана активно развиваются методы математического моделирования распространения звука в трёхмерных неоднородных волноводах и разрабатываются комплексы программ, основанные на данных методах.
            \par Математическое описание таких эффектов как отражение, преломление, диффракция и рассеяние акустической энергии в трёхмерном пространстве, то есть одновременно в вертикальной и горизонтальной плоскостях, рассматриваются в литературе на протяжении  нескольких десятилетий. Обзоры наиболее ранних работ представлено, например, в книге \cite{lee1995} и статье \cite{tolstoy1996}. В дальнейшем были разработаны первые эффективные и достаточно точные методы численного моделирования распространения звука, например, основанные на трёхмерных параболических уравнениях \cite{smith1999,sturm2003,heaney2016}. Необходимость учитывать трёхмерные эффекты также была экспериментально подтверждена в ряде работ 1990-2005 гг. \cite{tolstoy1996,frank2005,heaney2009,lynch2010}.
            \par Со временем также были разработаны идеализированные модельные задачи, отражающие характерные гидрологические, батиметрические и геологические (связанные со структурой слоёв дна) особенности, создающие различные трёхмерные эффекты \cite{bradley1971,buckingham1987,deane1993}. Наиболее часто используемой является задача распространения звука в клиновидном прибрежном волноводе мелкого моря, которая на данный момент лежит в основе процесса валидации разрабатываемых методов и моделей \cite{lee2019a,lee2019b,petrov2019,porter2019,sagers2019a}. Однако существуют и другие модельные задачи, фокусирующиеся на исследовании влияния формы и параметров волновода на распространение звука, например, волноводы, сформированные подводными каньонами, поперечные сечения которых имеют форму функции Гаусса \cite{barclay2019,lee2019b}, а также искривлёнными фронтами внутренних волн \cite{milone2019}; волноводы, содержащие одиночные и двойные подводные возвышенности \cite{porter2019}, шепчущие галереи, формируемые за счёт рефракции над чашеобразным дном \cite{katsnelson2019}; волноводы с переменной скоростью звука \cite{decourcy2019}. 
            \par Немаловажно также отметить работы, посвящённые исследованию трёхменых эффектов возникающих в реальных волноводах, такие как различного рода явления, возникающие из-за соляных клинов в устьях рек и нелинейных внутренних волн в областях континентального шельфа, и влиянию батиметрии на трёхмерную фокусировку, расфокусировку и дифракцию акустических волн. Подобные эффекты были рассмотрены в речных устьях \cite{reeder2019}, озёрах \cite{sagers2019b}, в каньоне Хадсона \cite{ballard2019,barclay2019}, в Восточно-Китайском море \cite{porter2019}, в областях с выраженными нелинейными внутренними волнами \cite{dossot2019,duda2019}, на материковых склонах \cite{dall2019}, а также при рассеивании звука над Срединно-Атлантическим хребтом \cite{oliveira2019} и абиссальной равниной\cite{stephen2019}, и даже через скопления пузырьков, производимых горбатыми китами во время питания \cite{qing2019}.
            \par Большое внимание в литературе уделяется исследованию точности и применимости различных современных численных методов моделирования распространения и рассеивания звука в трёхмерных океанологических волноводах. Например, методы конечных элементы были рассмотрены в работе \cite{qing2019}, где они были применены для моделирования акустического давления в пузырьковых сетях, создаваемого пением горбатых китов. Также, их точность была оценена при моделировании распространения в воде и дне сейсмо-акустических волн, создаваемых землетрясениями \cite{lecoulant2019}. Другая группа численных методов, основанных на методе параболического уравнения и впервые представленных в подводной акустике работами \cite{hardin1973,tappert1974}, рассмотрена в таких работах как \cite{ivanson2019,lee2019a,lee2019b,lin2019}, в которых особое внимание уделено перекрёстным членам, содержащим производные по глубине и угловой координате и возникающим из квадратного корня при формальной факторизации уравнения Гельмгольца. Также, в работе \cite{lin2019}, описано использование вычислительной сетки, предназначенной для лучшей обработки граничных условий. Методы трассировки лучей также были применены для расчёта трёхмерных звуковых полей и протестированы в ходе моделирования натурных экспериментов в Восточно-Китайском море в работе \cite{porter2019}. Улучшению качества моделирования распространения звука в задачах с реальными внутренними волнами посвящена работа \cite{duda2019}, в рамках которой был разработан составной метод моделирования, включающий нелинейную модель внутренних волн в региональную модель, основанную на реальных данных и учитывающую приливы и отливы. Также, внимание уделяется моделированию рассеяния, например, с использованием методов, основанных на явной численной схеме для решения интегрального уравнения Кирхгофа во временной области \cite{chen2019}, методе кратковременных эквивалентных источников для задач широкополосного рассеяния \cite{fahnline2019}, численном методе расчёта функции Грина для вычисления рассеяния на больших расстояниях, возникающего из-за объектов, расположенных на морском дне или лежащих в нём.
            \par Многие методы, описанные в литературе, также имеют открытую программную реализацию. Так, например, программы BELLHOP3D \cite{porter2019} и TRACEO3D \cite{calazan2017} реализуют метод моделирования распространения звука, основанный на трассировке лучей и гауссовых пучков. В KRAKEN3D \cite{porter1992} реализовано вычисление акустических полей в рамках модового разложения. Также, моделирование путём решения трёхмерных параболических уравнений реализовано в CAPRE3D \cite{duda2006}.
            \par Настоящая диссертация посвящена широкоугольным модовым параболическим уравнениям, которые известны уже достаточно давно, однако не получили широкого распространения. Несмотря на это, использование таких уравнений является перспективным, так как в сравнении с узкоугольными параболическими уравнениями они позволяют получать более точные решения, при этом требуя лишь незначительно больше вычислений. Так, самым трудоёмким этапом является решение акустической спектральной задачи, которое может быть вычислено заранее для изучаемой области, что значительно ускоряет процесс моделирования, так как само решение уравнения занимает лишь несколько минут. В работе также изучено применение лучевых стартеров, которые являются более подходящими для решения широкоугольных параболически уравнений в сравнении с традиционно используемыми начальными условиями Гаусса и Грина.
%       \inlineparagraph{Степень разработанности темы исследования}
        \inlineparagraph{Цели и задачи диссертационной работы}
            Целью настоящей диссертации является разработка эффективного метода моделирования распространения широкополосных акустических сигналов в трёхмерном волноводе мелкого моря и комплекса программ на основе этого метода, позволяющего решать широкий класс задач за разумное время.
            \par В ходе работы на диссертацией были решены следующие задачи.
            \begin{itemize}
                \item Разработать эффективный метод численного решения широкоугольных параболических уравнений для моделирования поля точечного источника звука с возможностью искусственного ограничения расчётной области.
                \item Разработать комплекс программ на языке программирования C++, позволяющий выполнять моделирование распространения звука в волноводах, имеющих произвольную структуру, путём задания батиметрии, гидрологии и параметров слоёв дна.
                \item Выполнить всестороннюю валидацию комплекса программ путём решения серии модельных задач и его апробацию в ходе выполнения расчётов уровней антропогенных акустических шумов в океане.
            \end{itemize}
        \inlineparagraph{Научная новизна}
            В работе имеются следующие элементы научной новизны
            \begin{enumerate}
                \item Разработан новый алгоритм численного решения начально-краевых задач для псевдодифференциальных модовых параболических уравнений с граничными условиями прозрачности и начальным условием моделирующим точечный всенаправленный источник колебаний.
                \item Разработан новый комплекс программ на языке программирования C++, реализующий предложенный алгоритм численного решения, имеющий возможность указания параметров волновода с использованием конфигурационных файлов и размещённый в открытом доступе.
                \item Впервые выполнены расчёты поля широкополосного акустического шума сухогруза на заданной акватории с учётом трёхмерного характера распространения звука в мелком море с неоднородным рельефом дна.
            \end{enumerate}
        \inlineparagraph{Теоретическая и практическая значимость}
            \par В диссертации предложен метод моделирования распространения звука в волноводах мелкого моря с использованием модовых параболических уравнений. Предложенный метод реализован в виде комплекса программ на языке программирования C++ и рамещён в открытом доступе. Разработанная программа позволяет выполнять трёхмерное моделирование распространения звука, трассировку лучей, соответствующих вертикальным модам, вычисление временного ряда импульса звукового сигнала и уровня звуковой экспозиции. При этом параметры волновода, модели и вычислений задаются с использованием конфигурационного файла, что существенно упрощает проведение моделирования, сокращая время, затрачиваемое на подготовку вычислений. Также, при разработке комплекса программ, существенное внимание было уделено возможности её использования в качестве заголовочной библиотеки, таким образом реализованный метод решения параболического уравнения может быть интегрирован в другие программы на языке программирования C++.
%           \par Теоретическая значимость работы состоит в том
        \inlineparagraph{Методология и методы исследования}
            Алгоритм численного решения псведодифференциальных модовых параболических уравнений (ПДМПУ) является модификацией известного метода SSP (split-step Pad\'e, метод расщепления Паде), к которому добавлены граничные условия прозрачности для искусственного ограничения расчётной области. При дискретизации дифференциального оператора по поперечной переменной в методе SSP использованы аппроксимации по методу конечных разностей. При задании начальных условий для ПДМПУ используется лучевое представление акустического поля на небольшом расстоянии от точечного источника.
            \par Программный продукт был написан на языке программирования C++ с использованием стандарта языка c++20. При разработке особое внимание уделялось возможности использования реализации предложенных алгоритмов в качестве сторонней заголовочной библиотеки, путём широкого использования методов объектно-ориентированного и шаблонного программирования. Библиотека boost была использована для упрощения реализации интерфейса командной строки. Для вычисления дискретного преобразования Фурье использовалась библиотека fftw. Оптимизация операций, связанной с линейной алгеброй, была выполнена с применением библиотеки Eigen. Библиотека CAMBALA использовалась для вычисления модовых функций и соответствующих им волновых чисел. Автоматизация процесса сборки для разных платформ выполнена с использованием программного средства CMake.
        \inlineparagraph{Положения, выносимые на защиту}
            \begin{enumerate}
                \item Разработана и апробирована методика численного решения псевдодифференциальных модовых параболических уравнений с искусственным ограничением расчётной области путём постановки граничных условий прозрачности или добавления к ней согласованных поглощающих слоёв.
                \item Разработан комплекс программ на языке программирования C++, который может быть использован для моделирования распространения тональных и импульсных сигналов, а также вычисления скалярных и векторных акустических полей антропогенных шумов в океане с возможностью учёта батиметрических и гидрологических данных и структуры слоёв дна, и ориентированный на максимальную производительность.
                \item Моделирование антропогенных шумов, связанных с сейсморазведочными работами и судоходством, проведённое с использованием разработанного комплекса прикладных программ, позволило добиться согласия уровней акустической экспозиции (SEL) с данными прямых измерений с точностью до 1 дБ. и согласия значений распределения энергии в децидекадных частотных полосах на различных расстояниях от источника шума с точностью до 5 дБ. для диапазона частот, в котором сосредоточена большая часть энергии сигнала.
            \end{enumerate}
        \inlineparagraph{Степень достоверности и апробация результатов}
            Методы, описанные в работе, а также их программная реализация, были протестированы на множестве модельных задач и на экспериментах с использованием данных подводных акустических зондов. Достоверность результатов обусловливается хорошей их согласованностью с известными методами моделирования и результатами натурных измерений. Основные результаты диссертации докладывались на следующих конференциях: на конференции <<Days on Diffraction>> (Санкт-Петербург, 2019, дистанционно 2022), на сессиях Российского акустического общества (дистанционно 2022, 2023), на конференции <<Океанологические исследования>> (Владивосток, 2023).
        \inlineparagraph{Публикации}
            Материалы диссертации опубликованы в 11 печатных работах, из них 5 в в рецензируемых научных журналах, включённых в перечень ВАК, а также индексируемых в международных библиографических базах данных Scopus ("Скопус") и Web of Science ("Сеть науки").
        \inlineparagraph{Личный вклад автора}
            Автор работы принмал участие в разработке численных методов решения модовых параболических уравнений, в том числе в областях с открытыми границами. Автором работы была проведена программная реализация численных методов решения уравнения Гельмгольца с использованием широкоугольных параболических уравнений. Также он принимал участие в разработке, реализации и апробации методики учёта данных опорных измерений при моделировании натурных акустических экспериментов.
        \inlineparagraph{Структура и объём диссертации} Диссертация состоит из введения, 4 глав, заключения и списка литературы.
            \edef\bibpage{\getpagerefnumber{sec::bibliography}}
            \edef\bibpages{\number\numexpr\getpagerefnumber{sec::bibliography_end}-\bibpage+1\relax}
            \edef\textpages{\number\numexpr\bibpage-\getpagerefnumber{sec::introduction}\relax}
            \par Общий объём диссертации \getpagerefnumber{LastPage} страниц, из них \textpages{} страниц текста, включая \totalfigures{} рисунка и \totaltables{} таблиц. Список литературы включает \total{citenum} наименований на \bibpages{} страницах.
%       \inlineparagraph{Благодарности}
\end{document}
