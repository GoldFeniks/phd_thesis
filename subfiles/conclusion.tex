\documentclass[../document.tex]{subfiles}

\begin{document}
    \section*{Заключение}
    \par Таким образом в рамках проделанной работы были исследованы методы решения МПУ с использованием аппроксимации Паде произвольного порядка, лучевых начальных условий и PML граничных условий, разработана численная схема их решения, разработан комплекс программ на языке C++, реализующий полученную численную схему с использованием пакета CAMBALA и возможностью вычисления временного ряда импульса звукового сигнала, значения распределения уровней SEL и координат распространения лучей, соответствующих вертикальным модам; проведены различные вычислительные эксперименты и изучена корректность и применимость полученного метода в сравнении с ШМПУ и другими методами решения уравнения Гельмгольца. В настоящее время комплекс программ уже используется сотрудниками лаборатории 2/4 ТОИ ДВО РАН для моделирования и оценки влияния антропогенных акустических шумов на шельфе о. Сахалин.
    \par Результаты работы были представлены на различных научных мероприятиях, включая PACON, UACE, Days on Diffraction \cite{dd}. Часть результатов работы была опубликована в \guillemotleft Journal of Sound and Vibration\guillemotright\ \cite{jsv}, а также направлена статья в \guillemotleft Акустический журнал\guillemotright\ \cite{acoustic_journal2021}, сфокусированная на разработанном программном продукте, в настоящий момент статья принята к печати и выйдет в конце 2021 г.
\end{document}
