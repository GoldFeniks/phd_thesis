\documentclass[../document.tex]{subfiles}

\begin{document}
    \section*{Заключение}
    \par Основными результатами, полученными в настоящем диссертационном исследовании, являются:
    \begin{itemize}
        \item Разработана и апробирована методика численного решения псевдодифференциальных модовых параболических уравнений с искусственным ограничением расчётной области путём постановки граничных условий прозрачности или добавления к ней согласованных поглощающих слоёв.
        \item Разработан комплекс программ на языке программирования C++, который может быть использован для моделирования распространения тональных и импульсных сигналов, а также вычисления скалярных и векторных акустических полей антропогенных шумов в океане с возможностью учёта батиметрических и гидрологических данных и структуры слоёв дна, и ориентированный на максимальную производительность.
        \item Моделирование антропогенных шумов, связанных с сейсморазведочными работами и судоходством, проведённое с использованием разработанного комплекса прикладных программ, позволило добиться согласия уровней акустической экспозиции (SEL) с данными прямых измерений с точностью до 1 дБ и согласия значений распределения энергии в децидекадных частотных полосах на различных расстояниях от источника шума с точностью до 5 дБ для диапазона частот, в котором сосредоточена большая часть энергии сигнала.
    \end{itemize}
\end{document}
