\documentclass[../document.tex]{subfiles}

\begin{document}
    \section{Реализация метода расчёта акустических полей и его верификация на модельных задачах\label{sec::ample}}
        \par В текущей главе рассматривается программная реализация и валидация моделирования распространения звука на основе метода параболического уравнения, описанного в предыдущей главе. В рамках диссертационной работы был разработан комплекс программ AMPLE (Ample Modal ParaboLic Equations) на языке программирования C++, размещённый в открытом доступе \cite{ample} и позволяющий проводить моделирование распространения звука в сложных трёхмерных океанических волноводах. При разработке программы большое внимание было уделено возможности простого конфигурирования процесса вычисления для нового волновода без необходимости изменять исходный код программы. Так, основным входом программы является конфигурационный файл в формате JSON, содержащий информацию о параметрах среды, источника и вычислительной сетки.
        \par Основная часть главы посвящена всесторонней валидации разработанного программного комплекса и методов моделирования проводится на модельных примерах в сравнении с другими методами. В первую очередь сравнение производится с аналитическим решением для волновода с плоским дном с целью валидации корректности работы разработанных вычислительных схем. Дополнительно проводится исследование работы PML поглощающих слоёв. Далее рассматривается задача распространения звука в волноводе с подводным каньоном и исследуется влияние изменений батиметрии на результирующее поле. Следующий вычислительный эксперимент представляет собой моделирование распространения звуковых волн в мелком море с подводным клином, являющимся своего рода эталонным примером для валидации методов моделирования поля звукового давления. Решение, полученное с использованием AMPLE, сравнивается с решением методом изображений и ШМПУ \cite{bachelor}, показывающее важность использования высоких порядков аппроксимации оператора квадратного корня (см. \niceref{seq::pade}{Главу}). Также для данного волновода рассматриваются задачи трассировки горизонтальных лучей, моделирования векторного поля плотности потока энергии и распространения импульсных акустических сигналов. Последним рассматривается распространение звука в волноводе с близки к реальным батиметрией и профилем скорости звука в воде.
        \subsection{Описание программной реализации метода расчёта акустических полей}
            \par В рамках данного диссертационного исследования была разработана программа, реализующая методы моделирования звука, описание которых было приведено в Главе \ref{seq::math}. Исходный код программы расположен в открытом доступе по адресу: \url{https://github.com/GoldFeniks/Ample}. В рамках данной работы было сделано 162 коммита, добавлено 18028 и удалено 8505  строк кода на языке программирования C++.
            \subsubsection{Средства реализации}
                \par В качестве основного языка программирования для реализации проекта был выбран язык C++20 \cite{c++}, ввиду следующих соображений:
                \begin{enumerate}
                    \item C++ является современным кроссплатформенным развивающимся языком программирования;
                    \item С++ позволяет проводить эффективное управление памятью;
                    \item C++ поддерживает параллельные вычислению;
                    \item Программы, написанные на языке C++, зачастую работают быстрее аналогичных программ, написанных на других языках программирования;
                    \item Стандартная библиотека C++ имеет широкий функционал;
                    \item Для C++ существует большой набор библиотек, упрощающих разработку;
                    \item Пакет CAMBALA \cite{cambala} написан на C++, что упрощает его интеграцию.
                \end{enumerate}
            \subsubsection{Требования к аппаратному обеспечению}
                \begin{itemize}
                    \item Процессор Intel Core i7 2.5 ГГц;
                    \item 8 Гб оперативной памяти.
                \end{itemize}
            \subsubsection{Требования к программному обеспечению}
                \begin{itemize}
                    \item ОС Linux 5.12.2 и выше, и другие ОС, для которых существует компилятор языка C++20 \cite{c++};
                    \item Библиотека Boost \cite{boost} версии 1.69 и выше;
                    \item Библиотека nlohmann/json \cite{nlohmann} версии 3.9.1 и выше;
                    \item Библиотека FFTW \cite{fftw3,fftw05} версии 3.3.9 и выше.
                \end{itemize}
            \subsubsection{Требования к пользователю}
                \begin{itemize}
                    \item Умение пользоваться командной строкой;
                    \item Понимание работы с текстовыми и бинарными файлами;
                    \item Знание формата JSON \cite{json};
                    \item Понимание предметной области.
                \end{itemize}
            \subsubsection{Используемые модули}
                \begin{itemize}
                    \item CAMBALA \cite{cambala} -- используется для вычисления модовых функций и соответствующих им горизонтальных волновых чисел (см. \ref{sec::sound_modes});
                    \item DORK \cite{dork} -- реализация обобщённого метода Рунге-Кутты и плотной выдачи \cite{dense}, используется при вычислении координат распространения лучей, соответствующих вертикальным модам (см. \ref{sec::horizontal_rays});
                    \item delaunay \cite{delaunay} -- реализация S-hull алгоритма триангуляции \cite{shull}, используется для интерполяции данных гидрологии на облаке точек;
                    \item zip \cite{zip} -- C++ реализация функции zip, аналогично языку программирования Python.
                    \item Eigen \cite{eigenweb} -- библиотека, содержащая различные инструменты для работы с элементами линейной алгебры: векторы, матрицы, численные методы и прочие алгоритмы.
                \end{itemize}
            \subsubsection{Структура проекта}
                \begin{itemize}
                    \item\code{boundary_conditions} -- содержит реализацию PML граничных условий с возможностью использования произвольной функции $\beta\pa{y}$;
                    \item\code{coefficients} -- функции для вычисления аппроксимаций Паде произвольного порядка;
                    \item\code{config} -- класс, предоставляющий интерфейс взаимодействия с конфигурационным файлом (см. конфигурационный файл);
                    \item\code{initial_conditions} -- класс для вычисления различных начальных условий: начальное условие Гаусса \eqref{eq::gauss}, начальное условие Грина \eqref{eq::greene}, обобщённый источник Гаусса \cite{jensen}, лучевые начальные условия (см. \ref{sec::ray_starters});
                    \item\code{modes} -- класс для многопоточного вычисления модовых функций и горизонтальных волновых чисел с использованием пакета CAMBALA \cite{cambala};
                    \item\code{rays} -- класс для вычисления распространения звуковых лучей;
                    \item\code{solver} -- класс, реализующий многопоточное вычисление решения задачи \eqref{eq::PDMPE}.
                \end{itemize}
                \paragraph{io}
                    \par Содержит инструменты для чтения и вывода данных.
                    \begin{itemize}
                        \item\code{convertors} -- функции для преобразования комплексных значений и точек в формат JSON и обратно;
                        \item\code{reader} -- функции чтения многомерных данных из текстовых и бинарных потоков с проверкой соответствия заданному описанию размерностей;
                        \item\code{writer} -- класс для вывода данных в текстовые и бинарные файлы.
                    \end{itemize}
                \paragraph{threads}
                    \par Содержит инструменты управления и взаимодействия с многопточными вычислениями.
                    \begin{itemize}
                        \item\code{buffer_manager} -- класс, позволяющий производить безопасные чтение и запись буферов данных из разных потоков выполнения программы;
                        \item\code{pool} -- класс, реализующий многопоточную очередь выполнения задач;
                        \item\code{task} -- класс, описывающий единичную задачу многопоточных вычислений.
                    \end{itemize}
                \paragraph{utils}
                    \par Содержит вспомогательные инструменты используемые остальными модулями программы.
                    \begin{itemize}
                        \item\code{assert} -- функции вывода ошибок при несоответствии заданным условиям;
                        \item\code{callback} -- набор инструментов для создания и комбинирования функций обратного вызова;
                        \item\code{comparators} -- функции сравнения произвольных элементов;
                        \item\code{concepts} -- описание концептов для упрощения типизации данных;
                        \item\code{differentiation} -- классы и функции для вычисления первой производной по трём точкам в одно-, двух- и трёхмерном случаях;
                        \item\code{dimensions} -- класс, описывающий измерения входных и выходных данных (см. описание входных данных);
                        \item\code{event} -- класс, описывающий интерфейс события, для упрощения взаимодействия с процессом вычисления решения;
                        \item\code{fft} -- класс, описывающий вещественное и комплексное быстрые преобразования Фурье, упрощающие использование библиотеки FFTW \cite{fftw3,fftw05};
                        \item\code{interpolation} -- реализует различные виды интерполяции \cite{interpolation,delaunay_interpolation}: линейная, билинейная, трилинейная, интерполяция Делонэ;
                        \item\code{join} -- функции конкатенации произвольного количества строк с произвольным разделителем;
                        \item\code{multi_optional} -- класс, описывающий кортеж элементов различных типов, при этом каждый элемент не обязательно содержится в коллекции, аналогично \code{std::optional} \cite{optional};
                        \item\code{object_descriptor} -- класс, описывающий тип и параметры некоторого объекта, использующийся в конфигурационном файле (см конфиг);
                        \item\code{progress_bar} -- класс, отображающий в консоли прогресс выполнения некоторого процесса;
                        \item\code{to_string_helper} -- класс, упрощающий преобразование различных типов данных в строковое представление;
                        \item\code{types} -- набор типов, используемых в программе;
                        \item\code{utils} -- набор вспомогательных функций различного назначения;
                        \item\code{verbosity} -- функции управления уровнем выводимой информации во время работы программы.
                    \end{itemize}
                \paragraph{main}
                    \par Содержит основную логику работы программы:
                    \begin{itemize}
                        \item функция \code{main}, являющаяся точкой входа в программу;
                        \item обработка аргументов командной строки с использованием библиотеки Boost \cite{boost};
                        \item чтение и обработка конфигурационных данных;
                        \item запуск вычислений и вывод результата.
                    \end{itemize}
            \subsubsection{Аргументы командной строки\label{sec::command_line_args}}
                \par Запуск программы производится из командной строки следующим образом\\
                \centerline{\code{./AMPLE [jobs] [options]}} Аргумент \code{jobs} задаёт список вычислений разделённых символом пробел, которые необходимо выполнить. Допустимыми значениями являются:
                \begin{itemize}
                    \item\code{solution} -- решение уравнения Гельмгольца \eqref{eq::3DH}, является значением по умолчанию;
                    \item\code{modes} --  модовые функции и волновые числа, используя пакет\\ CAMBALA \cite{cambala};
                    \item\code{init} -- начальные условия;
                    \item\code{rays} -- траектории распространения звуковых лучей;
                    \item\code{sel} -- интегральная характеристика SEL \eqref{eq::SEL};
                    \item\code{impulse} -- импульсы сигнала источника в приёмниках;
                    \item\code{acceleration} -- колебательные ускорения по пространственным координатам $x,y,z$.
                \end{itemize}
                При этом некоторые задачи зависят от других, в результате чего последние будут выполнены и без явно их указания в списке аргументов, однако результаты этих вычислений не будут сохранены. Аргумент \code{options} представляет собой произвольный набор опций программы, описание которых приведено далее.
                \paragraph{Основные опции}
                    \begin{itemize}
                        \item\code{-h [--help]} -- отображает информацию об аргументах командной строки и завершить выполнение;
                        \item\code{-v [--verbosity] arg} -- задаёт уровень информации, отображаемой во время работы программы, допустимые значения
                            \begin{itemize}
                                \item $0$ -- ничего не отображать (значение по умолчанию),
                                \item $1$ -- отображать время работы,
                                \item $2$ -- показывать прогресс выполнения задачи,
                                \item $3$ -- дополнительно к $2$ вывести краткую информацию о задаче и среде;
                            \end{itemize}
                        \item\code{-c [--config] path} -- задаёт путь к конфигурационному файлу, значение по умолчанию -- \code{"config.json"}.
                    \end{itemize}
                \paragraph{Опции вывода}
                    \begin{itemize}
                        \item\code{-o [--output] path} -- задаёт путь к папке для вывода результатов вычислений, значение по умолчанию -- \code{"output"};
                        \item\code{--row_step k} -- выводить каждую $k-\text{ую}$ вычисленную строку, значение по умолчанию -- $10$;
                        \item\code{--col_step k} -- выводить каждый $k-\text{ый}$ вычисленный столбец, значение по умолчанию -- $1$;
                        \item\code{--binary} -- использовать бинарный формат выходных файлов вместо текстового.
                    \end{itemize}
                \paragraph{Опции вычислений}
                    \begin{itemize}
                        \item\code{-w [--workers] arg} -- задаёт количество потоков, используемых для вычислений, значение по умолчанию -- $1$;
                        \item\code{-b [--buff] arg} -- задаёт размер буфера, используемого во время параллельных вычислений, значение по умолчанию -- $100$.
                    \end{itemize}
            \subsubsection{Формат входных данных}
                \par Основным входным файлом программы является конфигурационный файл в формате JSON \cite{json}, содержащий информацию о параметрах среды, сетки вычислительной области, координатах приёмника и источника, свойствах начальных условий и численного решения. Примеры конфигурационных файлов даны в \niceref{app::config_samples}{Приложении}.
                \paragraph{Типы данных}
                    \par В дополнение к стандартным типам JSON в конфигурационном файле используются следующие типы:
                    \begin{itemize}
                        \item\code{complex} -- комплексное число, представляемое одним из следующих способов
                            \begin{itemize}
                                \item вещественное число, мнимая часть равна $0$;
                                \item список, состоящий из двух вещественных чисел -- вещественная и мнимая часть числа соответственно;
                                \item JSON объект, имеющий два поля, значения которых -- вещественные числа: \code{real} -- вещественная часть числа, \code{imag} -- мнимая часть числа;
                            \end{itemize}
                        \item\code{point}\label{misc::point} -- произвольная точка в пространстве $\mathbb{R}^3$, представляемая одним из следующих способов
                            \begin{itemize}
                                \item список, состоящий из трёх вещественных чисел -- координаты точки $x, y, z$ соответственно;
                                \item JSON объект, имеющий три вещественных поля \code{x}, \code{y}, \code{z} -- координаты точки.
                            \end{itemize}
                    \end{itemize}
                    Типы табличных данных, содержащихся в файлах, указаны в \niceref{tbl::file_data_types}{Таблице}.
                    \begin{fefutable}{|C{3cm}|m{6cm}|m{6cm}|}{Типы табличных данных\label{tbl::file_data_types}}
                        \hline
                        Тип данных & \multicolumn{1}{c|}{Текстовый файл} & \multicolumn{1}{c|}{Бинарный файл}\\
                        \hline
                        \code{double} & вещественное число, понимаемое стандартной библиотекой языка C++ & IEEE double \cite{ieee_double}\\
                        \hline
                        \code{complex} & два числа типа \code{double}, разделённых символом пробел & 16 байт -- два числа типа \code{double}\\
                        \hline
                        \code{point} & три числа типа \code{double}, разделённых символом пробел & 24 байта -- три числа типа \code{double}\\
                        \hline
                    \end{fefutable}
                    \FloatBarrier
                \paragraph{Поля конфигурационного файла}
                    \par Простые поля конфигурационного файла (скалярные значения и списки) указаны в \niceref{tbl::simple_config_fiels}{Таблице}.
                    \begin{table}[h]
                        \centering
                        \caption{Простые поля конфигурационного файла\label{tbl::simple_config_fiels}}
                        \begin{tabular}{|C{4cm}|C{2cm}|C{1.75cm}|m{8cm}|}
                            \hline
                            Имя & Тип значения & Знач. по ум. & \multicolumn{1}{c|}{Описание}\\
                            \hline
                            \paramrow[-1]{mode_subset}{double}{подмножество вычисляемых мод}
                            \hline
                            \paramrow[2]{ppm}{integer}{количество точек на $1$ метр при вычислении мод}
                            \hline
                            \paramrow[3]{ord_rich}{integer}{порядок аппроксимации Ричардсона \cite{richardson}}
                            \hline
                            \paramrow[-1]{max_mode}{integer}{максимальное количество используемых мод}
                            \hline
                            \paramrow[0]{n_modes}{integer}{использовать такое количество мод}
                            \hline
                            \paramrow[1]{n_layers}{integer}{количество водяных слоёв}
                            \hline
                            \paramrow[0]{x0}{double}{\multirow{3}{8cm}{параметры вычислительной области по координате $x$}}
                            \paramrow{x1}{double}{}
                            \paramrow{nx}{integer}{}
                            \hline
                            \paramrow{y0}{double}{\multirow{3}{8cm}{параметры вычислительной области по координате $y$}}
                            \paramrow{y1}{double}{}
                            \paramrow{ny}{integer}{}
                            \hline
                            \paramrow{z0}{double}{\multirow{3}{8cm}{параметры вычислительной области по координате $z$}}
                            \paramrow{z1}{double}{}
                            \paramrow{nz}{integer}{}
                            \hline
                            \paramrow[0]{y_s}{double}{координата $y$ источника}
                            \hline
                            \paramrow{z_s}{double}{глубина источника}
                            \hline
                            \paramrowlistvalue[1.5]{bottom_rhos}{double[]}{значения плотностей слоёв дна}
                            \hline
                            \paramrowlistvalue[500]{bottom_layers}{double[]}{значения глубин слоёв дна}
                            \hline
                            \paramrowlistvalue[1700]{bottom_c1s}{double[]}{\multirow{2}{8cm}{значения скорости звука на верхней и нижней границах дна}}
                            \paramrowlistvalue[1700]{bottom_c2s}{double[]}{}
                            \hline
                            \paramrowlistvalue[0,0.5]{betas}{double[]}{значения параметра $\beta$ \eqref{eq::K}}
                            \hline
                            \paramrow[true]{complex_modes}{bool}{учитывать затухание (использовать мнимую часть коэффициента \eqref{eq::K})}
                            \hline
                            \paramrow[true]{const_modes}{bool}{считать, что моды не зависят от $x$}
                            \hline
                            \paramrow[false]{additive_depth}{bool}{глубины донных слоёв задаются относительно глубины дна}
                            \hline
                            \paramrow[-pi/4]{a0}{double}{\multirow{3}{8cm}{параметры вычислительной сетки по углу при вычислении звуковых лучей}}
                            \paramrow[pi/4]{a1}{double}{}
                            \paramrow[90]{na}{integer}{}
                            \hline
                            \paramrow[0]{l0}{double}{\multirow{3}{8cm}{параметры вычислительной сетки по натуральному параметру при вычислении звуковых лучей}}
                            \paramrow[4000]{l1}{double}{}
                            \paramrow[4001]{nl}{integer}{}
                            \hline
                            \paramrow{mnx}{integer}{\multirow{2}{8cm}{количество точек по осям $x$ и $y$ при вычислении мод}}
                            \paramrow{mny}{integer}{}
                            \hline
                        \end{tabular}
                    \end{table}
                    \begin{table}[h]
                        \centering
                        \caption*{\textit{Окончание \niceref{tbl::simple_config_fiels}{таблицы}}}
                        \begin{tabular}{|C{4cm}|C{2cm}|C{1.75cm}|m{8cm}|}
                            \hline
                            Имя & Тип значения & Знач. по ум. & \multicolumn{1}{c|}{Описание}\\
                            \hline
                            \paramrow[0.02]{tolerance}{double}{минимальное относительное значение функции источника}
                            \hline
                            \paramrow[0]{reference_index}{integer}{индекс опорного источника \eqref{eq::ref}}
                            \hline
                            \paramrowlistvalue[-1,-1]{sel_range}{double[]}{диапазон частот для SEL (см. \ref{sec::SEL})}
                            \hline
                            \paramrow[false]{sel_strict}{bool}{учитывать только частоты из диапазона \code{sel_range}}
                            \hline
                        \end{tabular}
                    \end{table}
                    \par Следующие поля конфигурационного файла задаются JSON объектами и строками
                    \begin{itemize}
                        \item\code{init} -- тип используемых начальных условий, допустимые значения:\\ \code{"gauss"} \eqref{eq::gauss}, \code{"greene"} \eqref{eq::greene}, \code{"ray_simple"} \eqref{eq::ray_simple}.
                        \item\code{tapering} -- параметры сглаживания начальных условий на границах области, задаются JSON объектом, описанным в \niceref{tbl::tapering}{Таблице}.
                        \begin{table}[h]
                            \centering
                            \caption{Описание значения поля \code{tapering}\label{tbl::tapering}}
                            \begin{tabular}{|C{3cm}|C{1.5cm}|C{2cm}|m{8.5cm}|}
                                \hline
                                \multicolumn{2}{|C{4.5cm}|}{Поле} & Тип значения & \multicolumn{1}{c|}{Описание} \\
                                \hline
                                \multicolumn{2}{|C{4.5cm}|}{\multirow{3}{*}{\code{type}}} & \multirow{3}{*}{\code{string}} & \code{"angeled"} -- сглаживает начальное условие в угловом интервале, является значением по умолчанию\\
                                \multicolumn{2}{|C{4.5cm}|}{} & & \code{"percentage"} -- сглаживает начальное условие на части сетки\\
                                \hline
                                \multirow{4}{*}{\code{parameters}} & \code{left} & \multirow{2}{*}{\code{double}} & \multirow{2}{10cm}{значения левого и правого\hspace{4cm} интервалов сглаживания}\\
                                & \code{right} & & \\
                                \cline{2-4}
                                & \code{value} & \code{double} & одно значение для \code{left} и \code{right}, по умолчанию -- $0.1$\\
                                \hline
                            \end{tabular}
                        \end{table}
                        \item\code{coefficients} -- параметры аппроксимации квадратного корня, задаются JSON объектом, описанным в \niceref{tbl::coefficients}{Таблице}.
                        \begin{table}[h]
                            \centering
                            \caption{Описание значения поля \code{coefficients}\label{tbl::coefficients}}
                            \begin{tabular}{|C{3cm}|C{1.5cm}|C{2cm}|m{8.5cm}|}
                                \hline
                                \multicolumn{2}{|C{4.5cm}|}{Поле} & Тип значения & \multicolumn{1}{c|}{Описание} \\
                                \hline
                                \multicolumn{2}{|C{4.5cm}|}{\multirow{4}{*}{\code{type}}} & \multirow{4}{*}{\code{string}} & \code{"wampe"} -- коэффициенты для метода \ref{sec::root_wampe} \\
                                \multicolumn{2}{|C{4.5cm}|}{} & & \code{"ssp"} -- коэффициенты для метода \ref{sec::root_ssp}, является значением по умолчанию\\
                                \hline
                                \multirow{3}{*}{\code{parameters}} & \code{n} & \multirow{3}{*}{\code{integer}} & степень полинома знаменателя, по умолчанию -- $1$\\
                                \cline{2-2}\cline{4-4}
                                & \code{m} & & степень полинома числителя, если опущено, используется значение \code{n}\\
                                \hline
                            \end{tabular}
                        \end{table}
                        \item\code{boundary_conditions} -- параметры граничных условий, задаются\\ JSON объектом, описанным в \niceref{tbl::boundary_conditions}{Таблице}.
                        \begin{table}[h]
                            \centering
                            \caption{Описание значения поля \code{boundary_conditions}\label{tbl::boundary_conditions}}
                            \begin{tabular}{|C{3cm}|C{2.25cm}|C{2cm}|m{7.75cm}|}
                                \hline
                                \multicolumn{2}{|C{5.25cm}|}{Поле} & Тип значения & \multicolumn{1}{c|}{Описание} \\
                                \hline
                                \multicolumn{2}{|C{5.25cm}|}{\code{type}} & \code{string} & \code{"pml"} -- PML граничные условия (см. \ref{sec::PML})\\
                                \hline
                                \multirow{3}{*}{\code{parameters}} & \code{width} & \code{integer} & ширина граничных условий в точках\\
                                \cline{2-4}
                                & \code{function} & \code{object} & описание функции $\beta\pa{\zeta}$ (см. \niceref{tbl::betay}{Таблицу})\\
                                \hline
                            \end{tabular}
                        \end{table}
                        \begin{table}[h]
                            \centering
                            \caption{Описание функции $\beta\pa{\zeta}$\label{tbl::betay}}
                            \begin{tabular}{|C{3cm}|C{1.5cm}|C{2cm}|m{8.5cm}|}
                                \hline
                                \multicolumn{2}{|C{4.5cm}|}{Поле} & Тип значения & \multicolumn{1}{c|}{Описание} \\
                                \hline
                                \multicolumn{2}{|C{4.5cm}|}{\multirow{3}{*}{\code{type}}} & \multirow{3}{*}{\code{string}} & \code{"cubic"} -- функция \eqref{eq::betay}\\
                                \multicolumn{2}{|C{4.5cm}|}{} & & \code{"tabular"} -- функция, заданная таблично на отрезке $\left[0, 1\right]$\\
                                \hline
                                \multirow{2}{*}{\code{parameters}} & \code{scale} & \code{double} & параметр масштаба $\beta_0$\\
                                \cline{2-4}
                                & \code{values} & \code{double[]} & значения функции\\
                                \hline
                            \end{tabular}
                        \end{table}
                        \FloatBarrier
                        \item\code{input_data}\label{misc::input_data} -- список входных многомерных данных, формат которых описан в \ref{sec::multidimensional_data}. Также может являться строкой, задающей путь к файлу JSON, содержащему объект в том же формате. Для восстановления значений в промежуточных точках используется соответствующая количеству измерений линейная интерполяция \cite{interpolation}. Допустимы следующие категории данных
                            \begin{itemize}
                                \item\code{bathymetry} -- значения глубин дна, имеет 2 измерения, $x$ и $y$ соответственно.
                                \item\code{hydrology} -- значения профилей скорости звука в воде, имеет 2 измерения, $z$ и $x$ соответственно, специальное значение -1 означает, что скорость звука в данной точке не известна, для получения значений в промежуточных точках используется интерполяция Делонэ \cite{delaunay_interpolation}.
                                \item\code{phi_s} -- значения модовых функций в источнике, имеет одно рваное измерение -- количество мод в зависимости от номера частоты, не может иметь значений координат.
                                \item\code{phi_j} -- значения модовых функций в среде, имеет 4 измерения: количество мод в зависимости от номера частоты, $x$, $y$, $z$; первое измерение является рваным и не может иметь значений координат.
                                \item\code{frequencies} -- одномерный список значений частот при которых нужно вычислить звуковое поле, не может иметь значений координат.
                                \item\code{receivers} -- одномерный список точек приёма в формате \hyperref[misc::point]{\code{point}}, измерение не может иметь значений координат.
                                \item\code{k0} -- вещественные значения волновых чисел источника, имеет одно рваное измерение -- количество мод в зависимости от номера частоты, не может иметь значений координат.
                                \item\code{complex_k0} -- комплексные значения волновых чисел источника, аналогично \code{k0}.
                                \item\code{k_j} -- вещественные значения волновых чисел в среде, имеет 3 измерения: количество мод в зависимости от номера частоты, $x$, $y$; первое измерение является рваным и не может иметь значений координат.
                                \item\code{complex_k_j} -- комплексные значение волновых чисел в среде, аналогично \code{k_j}.
                                \item\code{source_function} -- одномерный список значений функции источника или приёмника в зависимости от времени.
                                \item\code{source_spectrum} -- одномерный список комплексных значений спектра источника или приёмника в зависимости от частоты, не может быть одновременно указан с \code{frequences}.
                            \end{itemize}
                    \end{itemize}
                    \FloatBarrier
                \paragraph{Описание многомерных данных\label{sec::multidimensional_data}}
                    \par Многомерные данные описываются следующим JSON объектом
                    \begin{itemize}
                        \item\code{type} -- категория описываемых данных, например \code{"hydrology"};
                        \item\code{dimensions} -- описание размерностей данных, представляет собой список, состоящий из комбинации следующий значений
                            \begin{itemize}
                                \item вещественное число -- количество элементов измерения;
                                \item JSON объект, формат которого описан в \niceref{tbl::dimension}{Таблице}, поля \code{values} и \code{bound} не могут быть указаны одновременно, но могут отсутствовать, в этом случае координаты измерения считаются равномерно распределёнными на отрезке $\left[0, 1\right]$, также указание координат может быть недопустимо для некоторых измерений определённых категорий данных (например, количество приёмников);
                                \item список указанных выше значений, в этом случае измерение задаётся несколько раз, так называемое \guillemetleft рваное\guillemetright\  измерение (например количество мод в зависимости от номера частоты).
                                \begin{table}[h]
                                    \centering
                                    \caption{Описание измерения\label{tbl::dimension}}
                                    \begin{tabular}{|C{3cm}|C{2cm}|m{10cm}|}
                                        \hline
                                        Поле & Тип значения & \multicolumn{1}{c|}{Описание} \\
                                        \hline
                                        \code{n} & \code{integer} & количество элементов измерения\\
                                        \hline
                                        \multirow{3}{*}{\code{values}} & \code{double[]} & явные значения координат\\
                                        \cline{2-3}
                                         & \code{string} & путь к текстовому файлу, содержащему значения координат\\
                                        \hline
                                        \code{bounds} & \code{object} & описание равномерной сетки координат согласно \niceref{tbl::bounds}{Таблице}\\
                                        \hline
                                    \end{tabular}
                                \end{table}
                                \begin{table}[h]
                                    \centering
                                    \caption{Описание равномерной сетки координат\label{tbl::bounds}}
                                    \begin{tabular}{|C{3cm}|C{2cm}|m{10cm}|}
                                        \hline
                                        Поле & Тип значения & \multicolumn{1}{c|}{Описание} \\
                                        \hline
                                        \code{a} & \code{double} & левая граница интервала\\
                                        \hline
                                        \code{b} & \code{double} & правая граница интервала\\
                                        \hline
                                        \code{d} & \code{double} & шаг сетки, игнорируется во входных данных\\
                                        \hline
                                    \end{tabular}
                                \end{table}
                            \end{itemize}
                            \item\code{values} -- список значений многомерных данных, при этом каждое отдельное измерение может быть заменено на путь к файлу, содержащему данные;
                            \item\code{binary} -- являются файлы указанные в \code{value} бинарными.
                    \end{itemize}
                    \FloatBarrier
                    \newpage
                \paragraph{Описание выходных данных}
                    \par Выходные записываются в папку, указываемую аргументами командной строки (см. \ref{sec::command_line_args}), если такой папки не существует, она будет создана автоматически. В случае наличия конфликтующих файлов они будут перезаписаны. В результате работы программы будут созданы файлы \code{config.json} и \code{meta.json}, а также папки, содержащие многомерные выходные данные. Файл \code{config.json} содержит информацию из конфигурационного файла, указанного при запуске программы, а также значения параметров, которые были использованы во время работы, но не были явно указаны. Файл \code{meta.json} содержит следующие поля
                    \begin{itemize}
                        \item\code{command_line_arguments} -- аргументы командной строки переданные при запуске программы.
                        \item\code{computation_time} -- время выполнения в секундах.
                        \item\code{f} -- список частот, использованных во время вычислений.
                        \item\code{jobs} -- список выполненных вычислений.
                        \item\code{original_config_path} -- путь к конфигурационному файлу, указанный в аргументах командной строки.
                        \item\code{outputs} -- список описаний многомерных выходных данных, аналогично \hyperref[misc::input_data]{\code{input_data}}. Возможные следующие категории
                            \begin{itemize}
                                \item\code{phi_j}, \code{k_j}, \code{phi_s}, \code{k0}, \code{complex_k0} -- аналогично \hyperref[misc::input_data]{\code{input_data}};
                                \item\code{init} -- начальные условия, использованные во время вычислений, имеет 2 измерения: количество мод и $y$;
                                \item\code{rays} -- пары координат $\pa{x,y}$ распространения звуковых лучей, имеет 3 измерения: количество мод, $\alpha$, $l$ (см. \ref{sec::horizontal_rays});
                                \item\code{sel} -- значения SEL в вычислительной области, имеет 3 измерения: $x,y,z$;
                                \item\code{impulse} -- импульс источника вычисленный в точках приёма, имеет 2 измерения: номер приёмника и время (см. \ref{sec::impulse});
                                \item\code{solution} -- вычисленное звуковое поле, имеет 4 координаты: номер частоты и $x,y,z$;
                                \item\code{acceleration_x}, \code{acceleration_y}, \code{acceleration_z} -- колебательные ускорения по соответствующим пространственным координатам, имеют 4 координаты: номер частоты и $x,y,z$.
                            \end{itemize}
                    \end{itemize}
                    Данные, зависящие от частоты, будут сохранены в отдельных папках, содержащих файл \code{meta.json}, описывающий эти данные, и отдельные файлы с данными для каждой частоты, имеющие названия \code{frequency.(txt|bin)}, где \code{frequency} -- частота при которой были получены данные. Измерения, представляющие собой количество мод, имеют разный размер в зависимости от номера частоты, для которой выполняются расчёты.
        \subsection{Вычислительные эксперименты}
            \subsubsection{Волновод мелкого моря с плоским дном}
                \par Проверка корректности работы вычислительных схем и PML граничных условий была проведена на примере моделирования акустических волн в волноводе Пекериса, являющимся волноводом с постоянной глубиной дна, поэтому волновые числа постоянны, а модовые функции зависят только от глубины. В рамках такой задачи известно аналитическое решение, которое может быть записано как \cite{jensen}
                \begin{equation}
                    p\pa{x,y,z}=\frac{i}{4}\sum\limits_{j=1}^{\infty}\varphi_j\pa{z_s}\varphi_j\pa{z}H_0^{\pa{1}}\pa{k_j\sqrt{x^2+y^2}}\,,
                \end{equation}
                где $H_0^{\pa{1}}$ -- функция Ханкеля нулевого порядка первого рода \cite{hankel}. При проведении экспериментов были установлены следующие параметры
                \begin{itemize}
                    \item источник расположен на глубине $z_s=100\ \text{м}$;
                    \item глубина дна равна $200\ \text{м}$;
                    \item частота источника равна $f=25\ \text{Гц}$;
                    \item звуковое поле вычисляется на глубине $z=30\ \text{м}$ на равномерной сетке
                    \begin{equation}
                        \begin{array}{lll}
                            y_0=-4\ \text{км}\,,&y_1=4\ \text{км}\,,&n_y=8001\,,\\
                            x_0=50\ \text{м}&x_1=10\ \text{км}\,,&n_x=10001\,;
                        \end{array}
                    \end{equation}
                    \item ширина PML слоёв равнялась $500$ точек с обоих сторон вычислительной области, параметр масштаба равен $\beta_0=5$ (см. \ref{sec::PML});
                    \item использовались простые лучевые начальные условия \eqref{eq::ray_simple} с апертурой $\alpha\approx\pm 89.95^\circ$;
                    \item количество потоков вычисления $4$.
                \end{itemize}
                При таких параметрах среды имеются три захваченные (водные) моды. Звуковое поле было вычислено методом SSP (см. \ref{sec::root_ssp}) с использование разных порядков аппроксимации Паде, результаты вычислений показаны на \niceref{fig::pekeris}{Рисунке}. Сравнение проводилось с аналитическим решением и решением, полученным предыдущей версией программы с начальным условием Грина и аппроксимацией Клаербоута квадратного корня \cite{bachelor}. Из рисунка видно, что использование лучевых начальных условий и большего порядка аппроксимации позволяет добиться существенно лучших результатов по сравнению с обычным широкоугольным параболическим уравнением, при этом апертура решения является явно выраженной. При $p=17$ полученное решение почти не отличается от аналитического, что говорит о том, что численная схема позволяет получать решение любой апертуры. Использование метода Крэнка-Николсон (см. \ref{sec::root_wampe}) ожидаемо показывает такие же результаты, таким образом, свойства полученного решения не зависят от выбора численной схемы. Время выполнения программы указано в \niceref{tbl::pekeris_times}{Таблице}. Принцип работы поглощающих слоёв PML изображён на \niceref{fig::pekeris_pml}{Рисунке}, звуковое поле постепенно затухает при движении вглубь слоя. На \niceref{fig::pekeris_greene}{Рисунке} показаны результаты вычисления акустического поля с использованием начального условия Грина. Полученное таким образом решение значительно хуже сохраняет широкоугольные свойства уравнения, образуя численный шум в начале вычислительной области. Время, затраченное на проведение вычислений, отображено в \niceref{tbl::pekeris_times}{Таблице}.
                \begin{figure}[h]
                    \centering
                    \begin{subfigure}[t]{0.49\textwidth}
                        \centering
                        \includegraphics[width=\textwidth]{pekeris.pdf}
                        \caption{Аналитическое решение}
                    \end{subfigure}
                    \begin{subfigure}[t]{0.49\textwidth}
                        \centering
                        \includegraphics[width=\textwidth]{pekeris_wampe.pdf}
                        \caption{Решение ШМПУ}
                    \end{subfigure}
                    \hfill
                    \begin{subfigure}[t]{0.49\textwidth}
                        \centering
                        \includegraphics[width=\textwidth]{pekeris_n1.pdf}
                        \caption{SSP, $p=1$}
                    \end{subfigure}
                    \begin{subfigure}[t]{0.49\textwidth}
                        \centering
                        \includegraphics[width=\textwidth]{pekeris_n5.pdf}
                        \caption{SSP, $p=5$}
                    \end{subfigure}
                    \hfill
                    \begin{subfigure}[t]{0.49\textwidth}
                        \centering
                        \includegraphics[width=\textwidth]{pekeris_n9.pdf}
                        \caption{SSP, $p=9$}
                    \end{subfigure}
                    \begin{subfigure}[t]{0.49\textwidth}
                        \centering
                        \includegraphics[width=\textwidth]{pekeris_n17.pdf}
                        \caption{SSP, $p=17$}
                    \end{subfigure}
                    \caption{Акустическое поле (в дБ отн. 1 м.) в волноводе Пекериса на глубине $z=30\ \text{м.}$ с использованием лучевых начальных условий\label{fig::pekeris}}
                \end{figure}
                \begin{figure}[h]
                    \centering
                    \begin{subfigure}[t]{0.49\textwidth}
                        \centering
                        \includegraphics[width=\textwidth]{pekeris.pdf}
                        \caption{Аналитическое решение}
                    \end{subfigure}
                    \begin{subfigure}[t]{0.49\textwidth}
                        \centering
                        \includegraphics[width=\textwidth]{pekeris_n5_greene.pdf}
                        \caption{SSP, $p=5$}
                    \end{subfigure}
                    \hfill
                    \begin{subfigure}[t]{0.49\textwidth}
                        \centering
                        \includegraphics[width=\textwidth]{pekeris_n9_greene.pdf}
                        \caption{SSP, $p=9$}
                    \end{subfigure}
                    \begin{subfigure}[t]{0.49\textwidth}
                        \centering
                        \includegraphics[width=\textwidth]{pekeris_n17_greene.pdf}
                        \caption{SSP, $p=17$}
                    \end{subfigure}
                    \caption{Акустическое поле (в дБ отн. 1 м.) в волноводе Пекериса на глубине $z=30\ \text{м.}$ с использованием начального условия Грина\label{fig::pekeris_greene}}
                \end{figure}
                \begin{table}[h]
                    \centering
                    \caption{Время вычисления звукового поля в волноводе Пекериса\label{tbl::pekeris_times}}
                    \begin{tabular}{|C{3.5cm}|C{2.5cm}|C{3.5cm}|C{2.5cm}|}
                        \hline
                        Порядок аппроксимации & Время работы, с & Порядок аппроксимации & Время работы, с\\
                        \hline
                        1 & 4.6 & 9 & 15.029\\
                        \hline
                        5 & 9.624 & 17 & 25.928\\
                        \hline
                    \end{tabular}
                \end{table}
                \begin{figure}
                    \centering
                    \includegraphics[width=0.6\textwidth]{pekeris_pml_n13.pdf}
                    \caption{Акустическое поле (в дБ отн. 1 м.) в волноводе Пекериса на глубине $z=30\ \text{м.}$, слои PML отмечены красной пунктирной линией, ширина слоёв составляет $1000$ точек, порядок аппроксимации $p=13$\label{fig::pekeris_pml}}
                \end{figure}
                \FloatBarrier
            \subsubsection{Волновод мелкого моря с подводным каньоном}
                \par Следующим вычислительным экспериментом является моделирование распространения звуковых волн, создаваемых точечным источником в мелком море с подводным каньоном, схематическое изображение такого волновода изображено на \niceref{fig::canyon}{Рисунке}.
                Рельеф дна описывается функцией
                \begin{equation}
                    z=h\pa{y}=h_0+\Delta h\sech^2\pa{\sigma y}\,.
                \end{equation}
                \begin{figure}[h]
                    \centering
                    \includegraphics[width=0.5\textwidth]{canyon_transparent.png}
                    \caption{Схематическое изображение подводного каньона\label{fig::canyon}}
                \end{figure}
                \par В данном эксперименте были использованы следующие параметры
                \begin{equation}
                    \begin{array}{ccc}
                        h_0=50\ \text{м}\,, & \Delta h=5\ \text{м}\,, & \sigma=0.005\,,\\
                        z_s=10\ \text{м}\,, & p=11\,, & \alpha\approx\pm86^\circ\,,\\
                        x_0=50\ \text{м}\,, & x_1=30\ \text{км}\,, & n_x=30001\,,\\
                        y_0=-1\ \text{км}\,, & y_1=1\ \text{км}\,, & n_y=2001\,,
                    \end{array}
                \end{equation}
                при которых на частоте $f=120\ \text{Гц}$ имеются четыре захваченные моды. Звуковое поле было вычислено с порядком аппроксимации $p=11$, результаты вычислений отображены на \niceref{fig::canyon_field}{Рисунке}, время вычисления составило $22.791\ \text{с}$. На рисунках хорошо видно, как звук фокусируется в каньоне. Решение, полученное предыдущей версией программы почти не отличается от решения, полученного методом SSP, что подтверждает теорию о том, что в рамках данной задачи апертура уравнения играет незначительное роль. На \niceref{fig::canyon_compare}{Рисунке} полученное решение сравнивается с решением трёхмерного параболического уравнения \cite{isakson14,lin12,shtrum16}. Из рисунка видно, что решения достаточно сильно совпадают, как и в случае с широкоугольным параболическим уравнением \cite{bachelor}.
                \begin{figure}[h]
                    \centering
                    \includegraphics[width=0.9\textwidth]{canyon_compare.pdf}
                    \caption{Сравнение результатов вычисления акустического поля (в дБ отн. 1 м.) в мелком море с подводным каньоном при $y=0\ \text{км.}$ на глубине $z=10\ \text{м}$.\label{fig::canyon_compare}}
                \end{figure}
                \begin{figure}[h]
                    \centering
                    \begin{subfigure}[t]{0.75\textwidth}
                        \centering
                        \includegraphics[width=\textwidth]{canyon_wampe.pdf}
                        \caption{Решение ШМПУ, $z=10\ \text{м}$}
                    \end{subfigure}
                    \hfill
                    \begin{subfigure}[t]{0.75\textwidth}
                        \centering
                        \includegraphics[width=\textwidth]{canyon_n11.pdf}
                        \caption{SSP, $z=10\ \text{м}$}
                    \end{subfigure}
                    \hfill
                    \begin{subfigure}[t]{0.75\textwidth}
                        \centering
                        \includegraphics[width=\textwidth]{canyon_wampe_deep.pdf}
                        \caption{Решение ШМПУ, $z=55\ \text{м}$}
                    \end{subfigure}
                    \hfill
                    \begin{subfigure}[t]{0.75\textwidth}
                        \centering
                        \includegraphics[width=\textwidth]{canyon_n11_deep.pdf}
                        \caption{SSP, $z=55\ \text{м}$}
                    \end{subfigure}
                    \caption{Акустическое поле (в дБ отн. 1 м.) в клиновидном волноводе\label{fig::canyon_field}}
                \end{figure}
                \FloatBarrier
            \subsubsection{Клиновидный волновод мелкого моря}
                \par Следующий вычислительный эксперимент был проведён для моделирования распространения звуковых волн в мелком море с подводным клином. Схематическое изображение этого волновода дано на \niceref{fig::wedge}{Рисунке}.
                \begin{figure}[h]
                    \centering
                    \includegraphics[width=0.5\textwidth]{wedge_transparent.png}
                    \caption{Схематическое изображение подводного каньона\label{fig::wedge}}
                \end{figure}
                Рельеф дна задаётся функцией
                \begin{equation}
                    z=h\pa{y}=h_0+y\tan\alpha\,.
                \end{equation}
                \par Точечный источник расположен достаточно далеко от вершины клина на глубине $z=z_s$. Акустическое поле было вычислено на глубине $z=30\ \text{м}$ и были использованы следующие параметры
                \begin{equation}
                    \begin{array}{ccc}
                        z_s=100\ \text{м}\,,&h_0=200\ \text{м}\,,&\alpha\approx2.86^\circ\,,\\
                       f=25\ \text{Гц}\,,&p=13\,,&a=\pm89.95^\circ\,,\\
                       x_0=50\ \text{м}\,,&x_1=25\ \text{км}\,,&n_x=50001\,,\\
                       y_0=-3.32\ \text{км}\,,&y_1=3.32\ \text{км}\,,&n_y=13282\,.
                    \end{array}
                \end{equation}
                \par Результат вычислений показан на \niceref{fig::wedge_field}{Рисунке}, время работы программы составило $173.386\ \text{с}$. В целом решения ШМПУ и SSP похожи, однако даже на значительно расстоянии от источника заметна более широкая апертура SSP решения. Так как используемая модель не учитывает взаимодействия мод, звуковое поле обрезается, ввиду того, что при уменьшении глубины дна пропадают моды старших порядков. На \niceref{fig::wedge_comp}{Рисунке} показано сравнение решений при различных координатах $x$, из которого видно, что вблизи источника SSP решение не образует численного шума, а при отдалении от источника становится заметна более широкая апертура этого решения. Также было проведено сравнение с решением методом изображений \cite{deane,tang}. Сравнение результатов моделирования на \niceref{fig::wedge_cross_compare}{Рисунке} показывает, что при малых углах скольжения относительно изобаты интерференционная картина  решения в дальнем поле полностью определяется горизонтальной рефракцией звука и межмодовым взаимодействием можно пренебречь. Также, как показано на \niceref{fig::wedge_compare}{Рисунке}, решения всех методов почти совпадают, не смотря на адиабатическую природу модовых параболических уравнений, при этом большая апертура SSP метода ещё сильнее приближает решение к решению методом изображений вдали от источника.
                \begin{figure}[h]
                    \centering
                    \begin{subfigure}[t]{0.49\textwidth}
                        \centering
                        \includegraphics[width=\textwidth]{wedge_wampe.pdf}
                        \caption{Решение ШМПУ}
                    \end{subfigure}
                    \begin{subfigure}[t]{0.49\textwidth}
                        \centering
                        \includegraphics[width=\textwidth]{wedge_n13.pdf}
                        \caption{SSP}
                    \end{subfigure}
                    \hfill
                    \caption{Акустическое поле (в дБ отн. 1 м.) в мелком море с подводным клином при $z=30\ \text{м.}$\label{fig::wedge_field}}
                \end{figure}
                \begin{figure}[h]
                    \centering
                    \begin{subfigure}[t]{0.9\textwidth}
                        \centering
                        \includegraphics[width=\textwidth]{wedge_comp_1.pdf}
                        \caption{$x_c=1\ \text{км.}$}
                    \end{subfigure}
                    \hfill
                    \begin{subfigure}[t]{0.9\textwidth}
                        \centering
                        \includegraphics[width=\textwidth]{wedge_comp_9.pdf}
                        \caption{$x_c=9\ \text{км.}$}
                    \end{subfigure}
                    \hfill
                    \begin{subfigure}[t]{0.9\textwidth}
                        \centering
                        \includegraphics[width=\textwidth]{wedge_comp_17.pdf}
                        \caption{$x_c=17\ \text{км.}$}
                    \end{subfigure}
                    \hfill
                    \begin{subfigure}[t]{0.9\textwidth}
                        \centering
                        \includegraphics[width=\textwidth]{wedge_comp_25.pdf}
                        \caption{$x_c=25\ \text{км.}$}
                    \end{subfigure}
                    \caption{Сравнение результатов вычисления акустического поля с использованием метода широкоугольного параболического уравнения и метода SSP (AMPLE) (в дБ отн. 1 м.) в мелком море с подводным клином при $x=x_c$\label{fig::wedge_comp}}
                \end{figure}
                \begin{figure}[h]
                    \centering
                    \begin{subfigure}[t]{\textwidth}
                        \centering
                        \includegraphics[width=\textwidth]{wedge_cross_10.pdf}
                        \caption{$x_c=10\ \text{км.}$}
                    \end{subfigure}
                    \begin{subfigure}[t]{\textwidth}
                        \centering
                        \includegraphics[width=\textwidth]{wedge_cross_25.pdf}
                        \caption{$x_c=25\ \text{км.}$}
                    \end{subfigure}
                    \caption{Сравнение результатов вычисления акустического поля с использованием метода мнимых источников и метода SSP (AMPLE) (в дБ отн. 1 м.) в мелком море с подводным клином при $x=x_c\ \text{км}$.\label{fig::wedge_cross_compare}}
                \end{figure}
                \begin{figure}[h]
                    \centering
                    \begin{subfigure}[t]{\textwidth}
                        \centering
                        \includegraphics[width=\textwidth]{wedge_comp.pdf}
                    \end{subfigure}
                    \hfill
                    \begin{subfigure}[t]{\textwidth}
                        \centering
                        \includegraphics[width=\textwidth]{wedge_comp_close.pdf}
                    \end{subfigure}
                    \caption{Сравнение результатов вычисления акустического поля с использованием метода мнимых источников, метода широкоугольного параболического уравнения и метода SSP (AMPLE) (в дБ отн. 1 м.) в мелком море с подводным клином при $y=0\ \text{км}$.\label{fig::wedge_compare}}
                \end{figure}
                \FloatBarrier
                \paragraph{Трассировка горизонтальных лучей}
                    \par Рассмотрим задачу трассировки горизонтальных лучей в клиновидном волноводе (см. \niceref{sec::horizontal_rays}{Главу}). При распространении в среде, однородной по пространственной координате $x$, то есть при $k_j\pa{x,y}\equiv k_j\pa{y}$, справедлив закон Снеллиуса
                    \begin{equation}
                        k_j\pa{y}\cos\pa{\alpha_{j,\psi}\pa{y}}=C\,,
                    \end{equation}
                    где $\alpha_{j,\psi}\pa{y}$ -- угол скольжения луча относительно оси $x$, $\psi$ -- угол скольжения луча в источнике (см. Рис. \ref{fig:ray}), $C\in\mathbb{R}$ -- некоторая константа.
                    \begin{figure}[h]
                        \centering
                        \begin{tikzpicture}
                            \draw[very thick] (0,0)  arc (150:100:5);
                            \draw[->]  (-0.5,  0.71)  -- (4.25,0.71);
                            \draw[->]  (2.62, -0.29)  -- (2.62, 3.25);
                            \draw[<->] ( 0.5,  0.5)   -- (2.62, 0.5)  node[midway,below]{$dx$};
                            \draw[<->] (2.83,  0.71)  -- (2.83, 2.2)  node[midway,right]{$dy$};
                            \draw      ( 0.5,  0.25)  -- (0.5,  0.71);
                            \draw      (2.62,  2.2)   -- (3.08, 2.2);
                            \draw      (   1,  0.71) arc (0:24.5:1)   node[midway,above right=-0.25 and 0]{\small $\alpha_{j,i}\pa{y}$};

                            \node at (0.1,1) {\small $\pa{x,y}$};
                            \node[circle,fill,inner sep=1.5pt] at (0.5,0.71) {};
                        \end{tikzpicture}
                        \caption{\label{fig:ray} Схематичное изображение распространения горизонтального луча.}
                    \end{figure}
                    Тогда, положив $C=k_j\pa{0}\cos\pa{\psi}$, получим
                    \begin{gather}
                        \tan\pa{\alpha_{j,\psi}\pa{y}}=\frac{dy}{dx}\,,\\
                        1+\tan^2\pa{\alpha_{j,\psi}\pa{y}}=\frac{1}{\cos^2\pa{\alpha_{j,\psi}\pa{y}}}\,,\\
                        \frac{k_j\pa{y}}{C^2}=\frac{1}{\cos^2\pa{\alpha_{j,\psi}\pa{y}}}=1+\pa{\frac{dy}{dx}}^2\,,\\
                        \frac{dy}{dx}=\sqrt{\frac{k_j^2\pa{y}}{k_j^2\pa{0}\cos^2\pa{\psi}}-1}\,,\\
                        dx=\frac{dy}{\sqrt{\frac{k_j^2\pa{y}}{k_j^2\pa{0}\cos^2\pa{\psi}}-1}}\,.
                    \end{gather}
                    Рассмотрим луч, исходящий из точки $x\pa{0,\psi}=y\pa{0,\psi}=0$ и выпущенный под углом $\psi$, представим его в виде кривой $\left\{x\pa{s,\psi},y\pa{s,\psi}\right\}$ и получим представление координаты $x$ в виде криволинейного интеграла второго рода вдоль кривой $l^\psi: y\pa{s,\psi}$ (значения координаты $y$ распространения луча)
                    \begin{equation}\label{eq::ray_line_integral}
                        x\pa{S,\psi}=\int\limits_{l^\psi_S}\frac{\dd y\pa{s,\psi}}{\sqrt{\frac{k_j^2\pa{y\pa{s,\psi}}}{k_j^2\pa{0}\cos^2\pa{\psi}}-1}}=\int\limits_0^S\frac{y^\prime_t\pa{s,\psi}\dd s}{\sqrt{\frac{k_j^2\pa{y\pa{s,\psi}}}{k_j^2\pa{0}\cos^2\pa{\psi}}-1}}\,.
                    \end{equation}
                    Сравнение результатов моделирования с использованием программы AMPLE и численного интегрирования \eqref{eq::ray_line_integral} представлены на Рис. \ref{fig::rays}.
                    \begin{figure}
                        \centering
                        \begin{subfigure}{0.8\textwidth}
                            \includegraphics[width=\textwidth]{rays1.pdf}
                        \end{subfigure}
                        \begin{subfigure}{0.8\textwidth}
                            \includegraphics[width=\textwidth]{rays2.pdf}
                        \end{subfigure}
                        \begin{subfigure}{0.8\textwidth}
                            \includegraphics[width=\textwidth]{rays3.pdf}
                        \end{subfigure}
                        \caption{Сравнение трассировки лучей путём численного решения системы \eqref{eq::ray_hamilton} (сплошная кривая) и численного интегрирования \eqref{eq::ray_line_integral} (пунктирная кривая) для трёх распространяющихся мод.\label{fig::rays}}
                    \end{figure}
                \paragraph{Моделирование векторного поля плотности потока энергии}
                    \par Плотность потока энергии описывает скорость передачи энергии акустической волны через единицу площади перпендикулярно к направлению распространения волны и определяется как активная компонента вектора интенсивности \cite{shurov2003}
                    \begin{equation}
                        \mathbf{I}\pa{x,y,z}=\frac{1}{2}\left<\mathcal{R}\pa{p\pa{x,y,z,t}\overline{\mathbf{v}\pa{x,y,z,t}}}\right>\,,
                    \end{equation}
                    где $\left<\cdot\right>$ обозначает усреднение по интервалу времени, кратному одному периоду, $\overline{\pa{\cdot}}$ -- оператор комплексного сопряжения, $p\pa{x,y,z,t}$ -- скалярное поле акустического давления во временной области, $\mathbf{v}\pa{x,y,z,t}$ -- векторное поле колебательного ускорения. При рассмотрении тональных сигналов векторное поле плотности потока энергии в частотной области для фиксированной частоты $f$ может быть выражено как
                    \begin{equation}
                        \hat{\mathbf{I}}\pa{x,y,z,\omega}=\frac{1}{2}\mathcal{R}\pa{\hat{p}\pa{x,y,z,\omega}\overline{\hat{\mathbf{v}}\pa{x,y,z,\omega}}}\,,
                    \end{equation}
                    где $\omega=2\pi f$ -- циклическая частота. Вычисление колебательной скорости в частотной области описано в \niceref{seq::particle_acceleration}{Главе}.
                    \par Результаты моделирования векторного поля потока энергии отображены на \niceref{fig::wedge_energy_flux}{Рисунке}. Как видно из рисунка при распространении звука в клиновидном волноводе звуковая энергия стремится в сторону увеличения глубины дна. При этом наблюдается относительно высокая скорость передачи энергии после отражения от мелководной части клина.
                    \begin{figure}
                        \centering
                        \includegraphics[width=0.6\textwidth]{wedge_energy_flux.pdf}
                        \caption{Модуль и поток векторного поля плотности потока энергии на частоте 25 Гц в мелком море с подводным каньоном при $z=30$ м.\label{fig::wedge_energy_flux}}
                    \end{figure}
                \paragraph{Моделирование распространения импульсных акустических сигналов}
                    \par Рассмотрим задачу распространения импульсного акустического сигнала в клиновидном волноводе. Функция источника была выбрана следующим образом
                    \begin{equation}
                        f\pa{t} = A\beta_nH_n^{\pa{1}}\pa{\frac{t-t_0}{\sigma}}e^{-\pa{\frac{t-t_0}{\sigma}}^2}\,,
                    \end{equation}
                    где $A\in\mathbb{R}$ -- параметр масштаба, $t_0$ -- центральное время, $\sigma$ -- параметр ширины функции, $H_n^{\pa{1}}$ -- функция Ханкеля первого рода порядка $n$, а $\beta_n$ определено как
                    \begin{equation}
                        \beta_n=\begin{dcases}
                            \frac{k!}{n!}\,,&n=2k\,,\\
                            \frac{k!\sqrt{4k+3}}{\pa{2k}!\pa{4k+2}}\,,&n=2k+1\,.
                        \end{dcases}
                    \end{equation}
                    Такой вид функции позволяет получить выражение для центральной частоты сигнала как
                    \begin{equation}
                        f_c=\frac{1}{2\pi\sigma}\sqrt{2n+1}\,.
                    \end{equation}
                    При моделировании были использованы следующие параметры функции источника
                    \begin{equation}
                        \begin{array}{cc}
                            A=2000\,,&t_0=0.3\ \text{с}\,,\\
                            \sigma=0.021\,,&n=21\,,\\
                            \multicolumn{2}{c}{f_c=50\ \text{Гц}\,.}
                        \end{array}
                    \end{equation}
                    Графики функции и спектра источника отображены на \niceref{fig::wedge_source_function}{Рисунке}. При проведении моделирования был рассмотрен частотный диапазон $\left[13,100\right]$, в модовом разложении были использованы $32$ моды . Временной ряд импульса сигнала был вычислен в точке приёма, расположенной по координатам $x_r=25000\ \text{м}\,,y_r=0\ \text{м}\,,z_r=30\ \text{м}$. Сравнение результатов моделирования было проведено с решением методом изображений \cite{deane,tang}. Из результатов сравнения на \niceref{fig::impulse_compare}{Рисунке} видно, что решения показывают хорошую сходимость, при этом наибольшее отличие наблюдается в предвестнике, которое возникает из-за ограничений модового разложения при распространении на низких частотах на больших расстояниях.
                    \begin{figure}[h]
                        \centering
                        \begin{subfigure}{0.99\textwidth}
                            \includegraphics[width=\textwidth]{wedge_source_function.pdf}
                        \end{subfigure}
                        \begin{subfigure}{0.99\textwidth}
                            \includegraphics[width=\textwidth]{wedge_source_spectre.pdf}
                        \end{subfigure}
                        \caption{Функция и спектр источника при моделировании распространения импульсных сигналов в клиновидном волноводе мелкого моря. Красной пунктирной линией отмечены границы рассматриваемых частот \label{fig::wedge_source_function}}
                    \end{figure}
                    \begin{figure}[h]
                        \centering
                        \begin{subfigure}{\textwidth}
                            \includegraphics[width=\textwidth]{wedge_impulse.pdf}
                        \end{subfigure}
                        \begin{subfigure}{\textwidth}
                            \includegraphics[width=\textwidth]{wedge_impulse_spectre.pdf}
                        \end{subfigure}
                        \caption{Сравнение результатов вычисления временного ряда в приёмнике (сверху) и его спектра (снизу) при распространении импульсного акустического сигнала в клиновидном волноводе мелкого моря \label{fig::impulse_compare}}
                    \end{figure}
                    \FloatBarrier
            \subsubsection{Волновод с реальной батиметрией}
                \par В качестве последнего эксперимента было проведено моделирование распространения звуковых волн в волноводе с использованием реальных данных батиметрии, рельеф дна которого изображён на \niceref{fig::sakhalin}{Рисунке}.
                \begin{figure}[h]
                    \centering
                    \includegraphics[width=0.5\textwidth]{sakhalin_transparent.png}
                    \caption{Изображение волновода с реальной батиметрией\label{fig::sakhalin}}
                \end{figure}
                \begin{figure}[h]
                    \centering
                    \includegraphics[width=0.5\textwidth]{sound_profile.pdf}
                    \caption{Изображение профиля скорости звука\label{fig::sakhalin_sound_profile}}
                \end{figure}
                Также был использован профиль скорости звука в воде, изображённый на \niceref{fig::sakhalin_sound_profile}{Рисунке} и задаётся формулой
                \begin{equation}
                    c\pa{z}=-\frac{29}{1+e^{-\pa{\frac{12z}{21}-6}}}+1491\,.
                \end{equation}
                \par При проведении эксперимента было вычислено акустическое поле источника, расположенного на глубине $z=z_s=4\ \text{м}$, и были использованы следующие параметры
                \begin{equation}
                    \begin{array}{ccc}
                        f=150\ \text{Гц}\,,&p=11\,,&\alpha=\pm89.95^\circ\,,\\
                        x_0=50\ \text{м}\,,&x_1=7.5\ \text{км}\,,&n_x=7500\,,\\
                        y_0=-2\ \text{км}\,,&y_1=2\ \text{км}\,,&n_y=4001\,.
                    \end{array}
                \end{equation}
                Время работы программы составило $59.756\ \text{с}$. Результаты вычислений показаны на \niceref{fig::sakhalin_field}{Рисунке}, из которого видно, что решение ШМПУ существенно уступает решению, полученному методом SSP с большим порядком аппроксимации Паде. Также можно заметить, как звук фокусируется в области с большей глубиной.
                \begin{figure}[h]
                    \centering
                    \begin{subfigure}[t]{0.75\textwidth}
                        \centering
                        \includegraphics[width=\textwidth]{sakhalin_wampe_z4.pdf}
                        \caption{Решение ШМПУ, $z=4\ \text{м}$}
                    \end{subfigure}
                    \hfill
                    \begin{subfigure}[t]{0.75\textwidth}
                        \centering
                        \includegraphics[width=\textwidth]{sakhalin_n11_z4.pdf}
                        \caption{SSP, $z=4\ \text{м}$}
                    \end{subfigure}
                    \hfill
                    \begin{subfigure}[t]{0.75\textwidth}
                        \centering
                        \includegraphics[width=\textwidth]{sakhalin_wampe_z20.pdf}
                        \caption{Решение ШМПУ, $z=20\ \text{м}$}
                    \end{subfigure}
                    \hfill
                    \begin{subfigure}[t]{0.75\textwidth}
                        \centering
                        \includegraphics[width=\textwidth]{sakhalin_n11_z20.pdf}
                        \caption{SSP, $z=20\ \text{м}$}
                    \end{subfigure}
                    \caption{Акустическое поле (в дБ отн. 1 м.) в волноводе с реальной батиметрией\label{fig::sakhalin_field}}
                \end{figure}
                \FloatBarrier
            \subsubsection{Результаты вычислительных экспериментов}
                \par В результате вычислительных экспериментов было показано, что разработанная численная схема с применением аппроксимации Паде произвольного порядка и лучевых начальных условий работает корректно и может быть использована при моделировании распространения звука в сложных неоднородных океанических волноводах. При этом новый метод показывает значительно более гладкое и точное поле вблизи источника, а вдали от него учитывает волны исходящие под большими углами. Также в некоторых задачах более вычислительно затратные методы могут быть заменены разработанным методом, использование которого требует гораздо меньше вычислительных затрат.
        \subsection{Выводы к третьей главе}
            \par В данной главе диссертации представлена программная реализация математических методов, описанных во второй главе, а также приведены модельные примеры демонстрирующие корректность реализованных методов. Комплекс программ реализован на языке программирования C++ с использованием современных методов написания программ и библиотек. Программа AMPLE позволяет выполнять моделирование распространения звука в трёхмерных волноводах мелкого моря, расчёт временных рядов в точках приёма при распространении импульсных акустических сигналов и значений уровня звукового воздействия, трассировку лучей, соответствующих вертикальным модам, а также вычисление векторного поля колебательных ускорений. Конфигурация программы выполняется с использованием единственного текстового файла в формате JSON, содержащего параметры среды, батиметрии и гидрологии, сетки вычислительной области, координаты приёмников, свойства начальных условий и численного решения, параметры источника, такие как частота, функция или спектр, параметры вычисления нормальных мод или их предрасчитанные значения. Результаты работы программы выводятся в отдельную папку и содержат информацию о проделанных вычислениях и использованных параметрах, а также бинарные или тестовые файлы содержащие значения моделированных величин. Код программы размещён в открытом доступе, что упрощает его использование в различных проектах. Всё это позволяет существенно сократить время, требуемое для подготовки и интерпретации вычислений.
            \par Валидация комплекса программ была выполнена на примере различных модельных задач. Во-первых было установлено полное соответствие полученного решения аналитическому в случае волновода мелкого моря с плоским дном, а также показано влияние порядка аппроксимации Паде и различных начальных условий. Далее было проведено сравнение с решением методом виртуальных источников на примере распространения звука в волноводе мелкого моря с подводным каньоном. Было показана высокая согласованность решений, при этом время, затраченное на работу программы AMPLE составляет несколько десятков секунд. Следующим было проведено моделирование распространения звука в клиновидном волноводе мелкого моря. Сравнение решения, полученного с использованием программы AMPLE, проводилось с решением, полученным методом изображений. Было показано, что, даже не смотря на адиабатическую природу модового параболического уравнения, решения полностью совпадают. Последним было проведено моделирования распространения звука в волноводе с реальной батиметрией, с целью демонстрации влияния апертуры уравнения на получаемый результат. Таким образом, было показано, что разработанная программа показывает корректные результаты расчётов, при этом не требуя значительного подбора параметров моделирования, по сравнению с, например, методами, основанными на трассировке лучей и гауссовых пучков, и имея существенно быструю скорость вычислений, по сравнению с, например, методами трёмерного параболического уравнения.
            \par Разработанный комплекс программ также может быть улучшен путём расчёта модовых функций для диапазона глубин, нежели на горизонтальной сетке, что позволит далее сократить время вычислений в обширном классе задач. Проведение вычислений на графическом процессоре также позволит сократить время работы программы. Класс задач, подходящих для использования разработанной программой, может быть существенно расширен путём учета взаимодействия мод.
            \par Результаты третьей главы опубликованы в работах и представлены на конференциях \cite{acoustic_journal2021,acoustic_journal2023,dd,petrov2024_3}.
\end{document}
