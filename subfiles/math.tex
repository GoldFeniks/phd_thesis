\documentclass[../document.tex]{subfiles}

\begin{document}
	\section{Математические методы}
		\subsection{Модовое представление звукового поля точечного источника\label{sec::sound_modes}}
			\par Звуковое поле $p\pa{x,y,z}$ (где $z$ обозначает глубину, а  $x,y$ --- координаты горизонтальной плоскости), создаваемое точечным источником в трёхмерном волноводе мелкого моря, расположенным по координатам $x=y=0$, $z=z_s$ и имеющего частоту $f$, описывается трёхмерным уравнением Гельмгольца \cite{jensen}
			\begin{equation}\label{eq::3DH}
				\pa{\Delta + K^2\pa{x,y,z}}p\pa{x,y,z}=-\delta\pa{x}\delta\pa{y}\delta\pa{z-z_s}
			\end{equation}
			с граничными условиями
			\begin{align}
				p\bigr|_{z=0}&=0\,,\\
				p\bigr|_{z=h^{-}}&=p\bigr|_{z=h^{+}}\,,\\
				\frac{1}{\rho}\frac{\partial p}{\partial\mathbf{n}}\biggr|_{z=h^{-}}&=\frac{1}{\rho}\frac{\partial p}{\partial\mathbf{n}}\biggr|_{z=h^{+}}\,,\\
				p\bigr|_{z=H}&=0
			\end{align}
			где $\rho\pa{x,y,z}$ --- плотность среды, $H$ -- нижняя граница расчётной области, $\mathbf{n}$ --- нормаль к границе раздела слоёв среды, $h$ --- глубина этого раздела. Коэффициент $K\pa{x,y,z}$ определяется как
			\begin{equation}\label{eq::K}
				K\pa{x,y,z}=\frac{\omega}{c\pa{x,y,z}}\pa{1+i\eta\beta\pa{x,y,z}}\,,
			\end{equation}
			где $c\pa{x,y,z}$ --- скорость звука, $\omega=2\pi f$ --- циклическая частота, $\beta\pa{x,y,z}$ --- коэффициент затухания звука, $\eta=\nicefrac{1}{40\pi\log_{10}e}$. Рассматривая задачу в однородной среде, то есть $\rho\pa{x,y,z}=\rho\pa{z}$, $c\pa{x,y,z}=c\pa{z}$ и $\beta\pa{x,y,z}=\beta\pa{z}$, используя технику разделения переменных и представляя звуковое поле в виде
			\begin{equation}
				p\pa{x,y,z}=A\pa{x,y}\varphi\pa{z}\,,
			\end{equation}
			при $x^2+y^2>0$ и $z\ne z_s$ уравнение \eqref{eq::3DH} можно привести к виду
			\begin{multline}\label{eq::modal_pe}
				\frac{1}{A\pa{x,y}}\pa{\frac{\partial^2 A\pa{x,y}}{\partial x^2}+\frac{\partial^2 A\pa{x,y}}{\partial y^2}}+\\
				\frac{1}{\varphi\pa{z}}\pa{\frac{\dd^2\varphi\pa{z}}{\dd z^2}+K\pa{z}^2\varphi\pa{z}}=0\,.
			\end{multline}
			Вводя константу разложения $k^2$, уравнение \eqref{eq::modal_pe} можно разложить на два для $A\pa{x,y}$ и $\varphi\pa{z}$ соответственно
			\begin{gather}
				\frac{\partial^2 A\pa{x,y}}{\partial x^2}+\frac{\partial^2 A\pa{x,y}}{\partial y^2}-k^2A\pa{x,y}=0\label{eq::hre_simple}\,,\\
				\frac{\dd^2\varphi\pa{z}}{\dd z^2}+\pa{K\pa{z}^2-k^2}\varphi\pa{z}=0\label{eq::acoustic_spectral_problem}\,.
			\end{gather}
			Используя граничные условия для исходного уравнения Гельмгольца, можно выписать задачу Штурма-Лиувилля для \eqref{eq::acoustic_spectral_problem}
			\begin{equation}
				\begin{dcases}
					\frac{\dd^2\varphi\pa{z}}{\dd z^2}+K^2\pa{z}\varphi\pa{z}=k^2\varphi\pa{z}\,,\\
					\varphi\bigr|_{z=0}=0\,,\\
					\varphi\bigr|_{z=h^{-}}=\varphi\bigr|_{z=h^{+}}\,,\\
					\frac{1}{\rho}\frac{\dd \varphi}{\dd z}\bigr|_{z=h^{-}}=\frac{1}{\rho}\frac{\dd \varphi}{\dd z}\bigr|_{z=h^{+}}\,,\\
					\varphi\bigr|_{z=H}=0\,.
				\end{dcases}
				\label{eq::SLP}
			\end{equation}
			Такая задача имеет счётное множество решений $\varphi_j\pa{z}$, которые называются модовыми функциями (или модами), с соответствующими им горизонтальными волновыми числами $k_j$. Такие решения обладают следующими свойствами
			\begin{enumerate}
				\item без учёта затухания ($\beta\pa{z}\equiv0$) все волновые числа являются вещественными, с учётом -- комплексными,
				\item без учёта затухания модовые функции образуют ортогональный базис в пространстве $L^2_{\rho}\left[0,H\right]$ со скалярным произведением $\mathcats{U}\cdot,\cdot\mathcats{T}$ определённым формулой
				%                
				\begin{equation}
					\stretchleftright{\mathcats{U}}{f,g}{\mathcats{T}}_\rho=\int\limits_{0}^{H}\frac{f\pa{z}g\pa{z}}{\rho\pa{z}}\dd z\,,
				\end{equation}
				\item функции $\varphi_j\pa{z,x,y}$ имеют ровно $j+1$ корень на отрезке $\left[0,H\right]$,
				\item имеет место неравенство
				\begin{equation}
					k_1^2>k_2^2>k_3^2>\dots\,,
				\end{equation}
				\item модовые функции ортогональны и нормированы
				\begin{equation}
					\int\limits_0^{H}\frac{\varphi_i\pa{z}\varphi_j\pa{z}}{\rho\pa{z}}\operatorname{dz}=\delta_{i,j}\,.
				\end{equation}
			\end{enumerate}
			\par Так как изменение параметров среды по координатам $x, y$ в задачах подводной акустики является существенно более медленным по сравнению с изменением по $z$, решение для неоднородной среды может быть поучено путём рассмотрения модовых функций локально при фиксированных значениях $x,y$, то есть искать решение задачи \eqref{eq::SLP} отдельно в каждой точке $x,y$. Таким образом, звуковое поле $p\pa{x,y,z}$ может быть представлено в виде модового разложения
			\begin{equation}\label{eq::MD}
				p\pa{x,y,z}=\sum\limits_{j=1}^\infty A_j\pa{x,y}\varphi_j\pa{z,x,y}\,,
			\end{equation}
			при этом $x,y$ в $\varphi_j\pa{z,x,y}$ понимаются как параметры функции. С использованием этого разложения \eqref{eq::3DH} преобразуется к следующему виду
			\begin{multline}
				\sum\limits_{j=1}^\infty\left(\frac{\partial^2 A_j\pa{x,y}}{\partial x^2}+2\frac{\partial A_j\pa{x,y}}{\partial x}\frac{\partial\varphi_j\pa{z,x,y}}{\partial x}+\frac{\partial^2\varphi_j\pa{z,x,y}}{\partial x^2}\right.+\\
				\frac{\partial^2 A_j\pa{x,y}}{\partial y^2}+2\frac{\partial A_j\pa{x,y}}{\partial y}\frac{\partial\varphi_j\pa{z,x,y}}{\partial y}+\frac{\partial^2\varphi_j\pa{z,x,y}}{\partial y^2}+\\
				k_j^2\pa{x,y}A_j\pa{x,y}\varphi_j\pa{z,x,y}\biggr)=-\delta\pa{x}\delta\pa{y}\delta\pa{z-z_s}\,.
			\end{multline}
			Уравнение для одной моды $j$ теперь может быть получено умножением в смысле скалярного произведения на $\varphi_j\pa{z,x,y}$ (то есть применением оператора $\int\limits_{0}^{H}\pa{\cdot}\frac{\varphi_j\pa{z,x,y}}{\rho\pa{x,y,z}}\dd z$)
			\begin{multline}
				\frac{\partial^2 A_j\pa{x,y}}{\partial x^2}+\frac{\partial^2 A_j\pa{x,y}}{\partial y^2}+k_j^2\pa{x,y}A_j\pa{x,y}+\\
				\sum\limits_{\ell=1}^\infty\mathcal{U}_{j,\ell}A_{\ell}\pa{x,y}+2\sum\limits_{\ell=1}^\infty\mathcal{V}_{j,\ell}\frac{\partial A_{\ell}\pa{x,y}}{\partial x}+2\sum\limits_{\ell=1}^\infty\mathcal{W}_{j,\ell}\frac{\partial A_{\ell}\pa{x,y}}{\partial y}=\\
				-\delta\pa{x}\delta\pa{y}\varphi_j\pa{z_s,x,y}\,,
			\end{multline}
			где матрицы $\mathcal{U}$, $\mathcal{V}$, $\mathcal{W}$ определяются как
			\begin{align}
				\mathcal{U}_{j,\ell}&=\int\limits_{0}^H\pa{\frac{\partial^2\varphi_{\ell}\pa{z,x,y}}{\partial x^2}+\frac{\partial^2\varphi_{\ell}\pa{z,x,y}}{\partial y^2}}\frac{\varphi_j\pa{z,x,y}}{\rho\pa{x,y,z}}\dd z\,,\\
				\mathcal{V}_{j,\ell}&=\int\limits_{0}^H\frac{\partial\varphi_\ell\pa{z,x,y}}{\partial x}\frac{\varphi_j\pa{z,x,y}}{\rho\pa{x,y,z}}\dd z\,,\\
				\mathcal{V}_{j,\ell}&=\int\limits_{0}^H\frac{\partial\varphi_\ell\pa{z,x,y}}{\partial y}\frac{\varphi_j\pa{z,x,y}}{\rho\pa{x,y,z}}\dd z\,.
			\end{align}
			Эти матрицы описывают межмодовое взаимодействие при распространении звука. Во многих классах задач этим взаимодействием можно пренебречь, что в результате приводит уравнение к следующему виду
			\begin{equation}\label{eq::HRE}
				\frac{\partial^2 A_j\pa{x,y}}{\partial x^2}+\frac{\partial^2 A_j\pa{x,y}}{\partial y^2}+k_j^2\pa{x,y}A_j\pa{x,y}=-\delta\pa{x}\delta\pa{y}\varphi_j\pa{z_s,0,0}\,.
			\end{equation}
			Такое уравнение принято называть уравнением горизонтальной рефракции. 
		\subsection{Модовые параболические уравнения}
			\par Для получения параболических аппроксимаций уравнение горизонтальной рефракции \eqref{eq::HRE} представляется в виде
			\begin{equation}
				\pa{\frac{\partial}{\partial x}+i\sqrt{k_j^2\pa{x,y}+\frac{\partial^2}{\partial y^2}}}\pa{\frac{\partial}{\partial x}-i\sqrt{k_j^2\pa{x,y}+\frac{\partial^2}{\partial y^2}}}A_j\pa{x,y}=0\,,
			\end{equation}
			и выделяется его решение, состоящее из волн, распространяющихся в положительном направлении оси $x$
			\begin{equation}
				\pa{\frac{\partial}{\partial x}-i\sqrt{k_j^2\pa{x,y}+\frac{\partial^2}{\partial y^2}}}A_j\pa{x,y}=0\,.
			\end{equation}
			Вводя модовое опорное волновое число $k_{j,0}$ и выделяя главную осцилляцию из $A_j\pa{x,y}$
			\begin{equation}
				A_j\pa{x,y}=e^{k_{j,0}x}\mathcal{A}_j\pa{x,y}\,,\nonumber
			\end{equation}
			получим задачу Коши для псевдодифференциального модового параболического уравнения
			\begin{equation}\label{eq::PDMPE}
				\begin{dcases}
					\frac{\partial\mathcal{A}_j\pa{x,y}}{\partial x}=ik_{j,0}\pa{\sqrt{1+L_j}-1}\mathcal{A}_j\pa{x,y}\,,\\
					\mathcal{A}_j\pa{0,y}=\mathcal{A}_{j,0}\pa{y}
				\end{dcases}
			\end{equation}
			где 
			\begin{equation}
				k_{j,0}^2L_j=\frac{\partial^2}{\partial y^2}+k_j^2\pa{x,y}-k_{j,0}^2\,.\nonumber
			\end{equation}
		\subsection{Аппроксимация Паде}
			\par Для решения уравнения \eqref{eq::PDMPE} необходимо выполнить линеаризацию оператора квадратного корня с использованием аппроксимации Паде. Пусть есть некоторая функция $F\pa{\lambda}$ тогда её аппроксимация может быть записана в виде
			\begin{equation}\label{eq::pade}
				F\pa{\lambda}\approx\mathcal{R}\pa{F,l,m}\pa{\lambda}\equiv\frac{P_{l,m}^F\pa{\lambda}}{Q_{l,m}^F\pa{\lambda}}\,,
			\end{equation}
			где $P_{l,m}^F\pa{\lambda},Q_{l,m}^F\pa{\lambda}$ обозначают многочлены порядка $l$ и $m$ соответственно \cite{jensen}. Коэффициенты многочленов могут быть получены приравниванием рациональной функции $\nicefrac{P_{l,m}^F\pa{\lambda}}{Q_{l,m}^F\pa{\lambda}}$ к разложению в ряд Тейлора функции $F\pa{\lambda}$, содержащей $l+m+1$ членов.
			\subsubsection{Аппроксимация оператора квадратного корня\label{sec::root_wampe}}
				\par Аппроксимация оператора квадратного корня может быть записана в виде
				\begin{equation}\label{eq::pade_root}
					ik_{j,0}\pa{\sqrt{1+L_j}-1}\approx\frac{P_{l,m}\pa{L_j}}{Q_{l,m}\pa{L_j}}\,,
				\end{equation}
				тогда
				\begin{equation}\label{eq::pade_mpe}
					\frac{\partial\mathcal{A}_j\pa{x,y}}{\partial x}=\frac{P_{l,m}\pa{L_j}}{Q_{l,m}\pa{L_j}}\mathcal{A}_j\pa{x,y}\,.
				\end{equation}
				Используя дискретизацию Крэнка-Николсон \cite{crank} на равномерной сетке $x=nh$, $\mathcal{A}_j^n\sim\mathcal{A}_j\pa{x_n,y}$, уравнение \eqref{eq::pade_mpe} в положительном направлении оси $x$ можно записать виде
				\begin{equation}\label{eq::CNMPE}
					D_h^+\mathcal{A}_j^n=\frac{P_{l,m}\pa{L_j}}{Q_{l,m}\pa{L_j}}\mathcal{A}_j^{\nicefrac{n+1}{2}}\,,
				\end{equation}
				где 
				\begin{align*}
					D_h^+=\mathcal{A}_j=\frac{\mathcal{A}_j^{n+1}-\mathcal{A}_j^n}{h}\,,&&\mathcal{A}_j^{\nicefrac{n+1}{2}}=\frac{\mathcal{A}_j^{n+1}+\mathcal{A}_j^n}{2}\,.
				\end{align*}
				Тогда, уравнение \eqref{eq::CNMPE} может быть преобразовано к виду
				\begin{equation}
					\mathcal{A}_j^{n+1}=\frac{U\pa{L_j}}{W\pa{L_j}}\mathcal{A}_j^n\,,
				\end{equation}
				где
				\begin{alignat*}{3}
					U\pa{L_j}&=-&hP_{l,m}\pa{L_j}-2Q_{l,m}\pa{L_j}\,,\\
					W\pa{L_j}&= &hP_{l,m}\pa{L_j}-2Q_{l,m}\pa{L_j}\,,
				\end{alignat*}
				многочлены степени $p=\max\pa{l,m}$. Положив $l\leqslant m$ и разложив отношение $\nicefrac{U\pa{L_j}}{W\pa{L_j}}$ на простые дроби, получим
				\begin{equation}\label{eq::dPDMPE}
					\mathcal{A}_j^{n+1}=\pa{a_{l,m}^0+\sum\limits_{i=1}^p\frac{a_{l,m}^i}{1+b_{l,m}^iL_j}}\mathcal{A}_j^n\,.
				\end{equation}
			\subsubsection{Метод аппроксимации Паде для пропагатора\label{sec::root_ssp}}
				\par Другой подход к решению \eqref{eq::PDMPE} был независимо предложен Авиловым \cite{avilov} и Коллинзом \cite{collins}. В его основе лежит смена порядка дискретизации и применения аппроксимации Паде. При достаточно маленьком интервале $\Delta x=h$ уравнение \eqref{eq::PDMPE} может быть формально решено в виде
				\begin{equation}
					\mathcal{A}_j^{n+1}=e^{ik_{j,0}h\pa{\sqrt{1+L}-1}}\mathcal{A}_j^n\,.
				\end{equation}
				Затем, применяя аппроксимацию Паде для экспоненты  в виде разложения на простые дроби аналогично аппроксимации квадратного корня
				\begin{equation}\label{eq::SSPADE}
					e^{ik_{j,0}h\pa{\sqrt{1+L}-1}}\approx\frac{\tilde{U}\pa{L_j}}{\tilde{W}\pa{L_j}}=\tilde{a}_{l,m}^0+\sum\limits_{i=1}^p\frac{\tilde{a}_{l,m}^i}{1+\tilde{b}_{l,m}^iL_j}\,,
				\end{equation}
				получим
				\begin{equation}\label{eq::dSSPADE}
					\mathcal{A}_j^{n+1}=\pa{\tilde{a}_{l,m}^0+\sum\limits_{i=1}^p\frac{\tilde{a}_{l,m}^i}{1+\tilde{b}_{l,m}^iL_j}}\mathcal{A}_j^n\,.
				\end{equation}
				Очевидно, что полиномы $\tilde{U}\pa{L_j},\tilde{W}\pa{L_j}$ отличаются от $U\pa{L_j},W\pa{L_j}$, также как и коэффициенты их разложения на простые дроби $\tilde{a}_{l,m}^i, \tilde{b}_{l,m}^i$ и $a_{l,m}^i,b_{l,m}^i$. Однако, в остальном уравнения \eqref{eq::dPDMPE} и \eqref{eq::dSSPADE} полностью идентичны. В дальнейшем символ $\tilde{\pa{\cdot}}$ будет опущен, так как все рассматриваемые методы могут быть применены к обоим формам. В англоязычной литературе такой метод имеет название Split-step Pad\'e.
			\subsubsection{Коэффициенты аппроксимации Паде\label{sec::PADE}}
				\par Пусть требуется вычислить аппроксимацию Паде некоторой функции $F\pa{\lambda}$ в виде \eqref{eq::pade}. Также, положим, что $l\leqslant m$, и будем искать разложение аппроксимации на простые дроби в виде
				\begin{equation}
					\frac{P_{l,m}^F\pa{\lambda}}{Q_{l,m}^F\pa{\lambda}}=\frac{\prescript{0}{}{\alpha}_{l,m}^F+\sum\limits_{i=1}^l\prescript{i}{}{\alpha}_{l,m}^F}{1+\sum\limits_{i=1}^m\prescript{i}{}{\beta}_{l,m}^F}=\prescript{0}{}{a}_{l,m}^F+\sum\limits_{i=1}^m\frac{\prescript{i}{}{a}_{l,m}^F}{1+\prescript{i}{}{b}_{l,m}^F\lambda}\,.
				\end{equation}
				\paragraph{Вычисление коэффициентов полиномов}
					\par Пусть известно разложение функции $F\pa{\lambda}$ в ряд Тейлора вида
					\begin{equation}
						F\pa{\lambda}\approx T_{l+m}^F\pa{\lambda}=c_0^F+\sum\limits_{i=1}^{l+m}c_i^F\lambda^n+o\pa{\lambda^{l+m}}\,,
					\end{equation}
					тогда
					\begin{equation}
						c_0^F+\sum\limits_{i=1}^{l+m}c_i^F\lambda^n=\frac{\prescript{0}{}{\alpha}_{l,m}^F+\sum\limits_{i=1}^l\prescript{i}{}{\alpha}_{l,m}^F}{1+\sum\limits_{i=1}^m\prescript{i}{}{\beta}_{l,m}^F}\,.
					\end{equation}
					Коэффициенты $\prescript{i}{}{\alpha}_{l,m}^F$ и $\prescript{i}{}{\beta}_{l,m}^F$ могут быть найдены из решения системы линейных алгебраических уравнений
					\begin{equation}
						\begin{alignedat}{4}
							\prescript{0}{}{\alpha}_{l,m}^F&=c_0^F\,,&&\\
							\prescript{1}{}{\alpha}_{l,m}^F&=c_1^F&+&\prescript{1}{}{\beta}_{l,m}^Fc_0^F\,,\\
							\prescript{2}{}{\alpha}_{l,m}^F&=c_2^F&+&\prescript{1}{}{\beta}_{l,m}^Fc_1^F+\prescript{2}{}{\beta}_{l,m}^Fc_0^F\,,\\
							\vdots\quad&\quad\ \ \vdots&&\quad\vdots\\
							\prescript{l}{}{\alpha}_{l,m}^F&=c_l^F&+&\sum\limits_{i=1}^l\prescript{i}{}{\beta}_{l,m}^Fc_{l-i}^F\,,\\
						\end{alignedat}\qquad
						\begin{alignedat}{4}
							0&=c_{l+1}^F&+&\sum\limits_{i=1}^{\min\pa{l+1,m}}\prescript{i}{}{\beta}_{l,m}^Fc_{l-i+1}^F\,,\\
							0&=c_{l+2}^F&+&\sum\limits_{i=1}^{\min\pa{l+2,m}}\prescript{i}{}{\beta}_{l,m}^Fc_{l-i+2}^F\,,\\
							\vdots&\quad\ \ \vdots&&\qquad\vdots\\
							0&=c_{l+m}^F&+&\quad\sum\limits_{i=1}^{m}\quad\prescript{i}{}{\beta}_{l,m}^Fc_{l-i+m}^F\,.
						\end{alignedat}
					\end{equation}
					Таким образом, $\prescript{i}{}{\alpha}_{l,m}^F$ явно выражены через коэффициенты $\prescript{i}{}{\beta}_{l,m}^F$, которые могут найдены из решения правой системы уравнений.
					\par В большинстве случаев выбор $l=m$ достаточно точно приближает аппроксимируемую функцию, однако в некоторых случаях использование $\theta$-комбинации \cite{chui,xavier} аппроксимаций с порядками полиномов $l=m$ и $l=m-1$ позволяет получить более точное решение. Пусть $\prescript{i}{}{\alpha}_{m,m}^F,\prescript{i}{}{\beta}_{m,m}^F$ и $\prescript{i}{}{\alpha}_{m-1,m}^F,\prescript{i}{}{\beta}_{m-1,m}^F$ коэффициенты соответствующих полиномов, тогда их $\theta$-комбинация при $\theta\in\left[0,1\right]$ может быть вычислена как
					\begin{equation}
						\begin{aligned}
							\prescript{i}{\theta}{\alpha}_{m,m}^F&=\theta\prescript{i}{}{\alpha}_{m,m}^F+\pa{1-\theta}\prescript{i}{}{\alpha}_{m-1,m}^F\,,\quad i=\overline{0,m-1}\,,\\
							\prescript{m}{\theta}{\alpha}_{m,m}^F&=\theta\prescript{m}{}{\alpha}_{m,m}^F\,,\\
							\prescript{i}{\theta}{\beta}_{m,m}^F&=\theta\prescript{i}{}{\beta}_{m,m}^F+\pa{1-\theta}\prescript{i}{}{\beta}_{m-1,m}^F\,,\quad i=\overline{1,m}\,.
						\end{aligned}
					\end{equation}
				\paragraph{Вычисление разложения на простые дроби}
					\par Пусть полиномы $P_{l,m}^F\pa{\lambda},Q_{l,m}^F\pa{\lambda}$ имеют только простые корни $\prescript{i}{}{p}_{l,m}^F$ и $\prescript{i}{}{q}_{l,m}^F$ соответственно, тогда разложение их частного на простые дроби методом Хэвисайда \cite{heavyside} может быть записано как
					\begin{equation}
						\begin{gathered}
							\frac{P_{l,m}^F\pa{\lambda}}{Q_{l,m}^F\pa{\lambda}}=\prescript{0}{}{a}_{l,m}^F+\sum\limits_{i=1}^m\frac{\prescript{i}{}{a}_{l,m}^F}{\lambda-\prescript{i}{}{q}_{l,m}^F}=\prescript{0}{}{a}_{l,m}^F+\sum\limits_{i=1}^m\frac{\prescript{i}{}{a}_{l,m}^F}{1+\prescript{i}{}{b}_{l,m}^F\lambda}\,,\\
							\prescript{i}{}{b}_{l,m}^F=-\pa{\prescript{i}{}{q}_{l,m}^F}^{-1}\,,\\
							\prescript{i}{}{a}_{l,m}^F=\prescript{i}{}{b}_{l,m}^F\frac{\prod\limits_{j=1}^l\prescript{i}{}{q}_{l,m}^F-\prescript{j}{}{p}_{l,m}^F}{\prod\limits_{\substack{j=1\\i\ne j}}^m\prescript{i}{}{q}_{l,m}^F-\prescript{j}{}{q}_{l,m}^F}\,,\\
							\prescript{0}{}{a}_{l,m}^F=\prescript{0}{}{\alpha}_{l,m}^F-\sum\limits_{i=1}^m\prescript{i}{}{a}_{l,m}^F\,.
						\end{gathered}
					\end{equation}
					Полиномы $P_{l,m}^F\pa{\lambda}$ и $Q_{l,m}^F\pa{\lambda}$ в общем случае имеют комплексные коэффициенты, корни которых являются собственными значениями матрицы компаньона \cite{numerical_recipes}
					\begin{equation}
						\begin{pmatrix}
							-\frac{a_m-1}{a_m} & -\frac{a_m-2}{a_m} & \cdots & -\frac{a_1}{a_m} & -\frac{a_0}{a_m}\\
							1 & 0 & \cdots & 0 & 0\\
							0 & 1 & \cdots & 0 & 0\\
							\vdots & \vdots & \ddots & \vdots & \vdots\\
							0 & 0 & \cdots & 1 & 0
						\end{pmatrix}\,,
					\end{equation}
					где $a_i$ коэффициенты полинома.
				\paragraph{Разложение в ряд Тейлора оператора квадратного корня и экспоненты}
				\par Коэффициенты разложения оператора квадратного корня\\
					$ik_{j,0}\pa{\sqrt{1+L_j}-1}$ могут быть выражены формулой
					\begin{equation}
						c_0=0\,,\quad c_p=ik_{0,j}\frac{\pa{\frac{1}{2}}_p}{p!},\quad p=\overline{1,n}\,,
					\end{equation}
					где $\pa{x}_p=\prod\limits_{k=0}^{p-1}\pa{x-k}$ -- убывающий факториал.
					\par Коэффициенты разложения экспоненты $e^{ik_{j,0}h\pa{\sqrt{1+L}-1}}$ могут быть получены из рекуррентного соотношения
					\begin{equation}
						\begin{aligned}
							c_0&=1\,,\\
							c_1&=\frac{ik_{j,0}h}{2}\,,\\
							c_p&=\frac{\pa{ik_{j,0}h}^2c_{p-2}-\pa{4p^2+6p+2}c_{p-1}}{4\pa{p+1}\pa{p+2}}\,,\quad p=\overline{2,n}\,.
						\end{aligned}
					\end{equation}
		\subsection{Дискретизация оператора $L_j$}
			\par Для дискретизации уравнения \eqref{eq::dPDMPE} на равномерной сетке $y_q=y_0+q\delta$ с шагом $\Delta y=\delta$ используется стандартная конечно-разностная схема
			\begin{equation}\label{eq::D2}
				D_\delta^2\mathcal{A}_j^{n+1}=\frac{\mathcal{A}_j^{n+1,q+1}-2\mathcal{A}_j^{n+1,q}+\mathcal{A}_j^{n+1,q-1}}{\delta^2}\,,\quad q\in\mathbb{N}\,,
			\end{equation}
			где $\mathcal{A}_j^{n,q}\sim\mathcal{A}_j\pa{x_n,y_q}$. Таким образом, уравнение \eqref{eq::dPDMPE} преобразуется к виду
			\begin{equation}\label{eq::ddPDMPE}
				\mathcal{A}_j^{n+1,q}=\pa{a_{l,m}^0+\sum\limits_{i=1}^p\frac{a_{l,m}^i}{1+b_{l,m}^iL_j^\delta}}\mathcal{A}_j^{n,q}\,,\quad q\in\mathbb{N}\,,
			\end{equation}
			где $k_{j,0}^2L_j^\delta=D_\delta^2+k_j^2-k_{j,0}^2$. Будем искать $\mathcal{A}_j^{n+1,q}$ в виде
			\begin{equation}\label{eq::BtoA}
				\mathcal{A}_j^{n+1,q}=a_{l,m}^0\mathcal{A}_j^{n,q}+\sum\limits_{i=1}^pa_{l,m}^i\mathcal{B}_{j,i}^{n+1,q}\,,
			\end{equation}
			тогда
			\begin{equation}
				a_{l,m}^0\mathcal{A}_j^{n,q}+\sum\limits_{i=1}^pa_{l,m}^i\mathcal{B}_{j,i}^{n+1,q}=\pa{a_{l,m}^0+\sum\limits_{i=1}^p\frac{a_{l,m}^i}{1+b_{l,m}^iL_j^\delta}}\mathcal{A}_j^{n,q}\,.
			\end{equation}
			Приравнивая слагаемые при одинаковом $i$, получим набор выражений, из которых могут быть найдены значения вспомогательных функций $\mathcal{B}_{j,i}^{n+1,q}$
			\begin{equation}\label{eq::BEQ}
				\pa{1+b_{l,m}^iL_j^\delta}\mathcal{B}_{j,i}^{n+1,q}=\mathcal{A}_j^{n,q}\,,\quad i=\overline{1,p}\,.
			\end{equation}
			Используя равенства \eqref{eq::D2} и $k_{j,0}^2L_j^\delta=D_\delta^2+k_j^2-k_{j,0}^2$, получим
			\begin{equation}
				\underset{\alpha_{j,i}}{\underbrace{\frac{b_{l,m}^i}{k_0^2\delta^2}}}\mathcal{B}_{j,i}^{n+1,q-1}+\underset{\beta_{j,i}}{\underbrace{\pa{1+\frac{b_{l,m}^i}{k_0^2}\pa{k_j^2-k_0^2-\frac{2}{\delta^2}}}}}\mathcal{B}_{j,i}^{n+1,q}+\underset{\gamma_{j,i}}{\underbrace{\frac{b_{l,m}^i}{k_{j,0}^2\delta^2}}}\mathcal{B}_{j,i}^{n+1,q-1}=\mathcal{A}_j^{n,q}\,.
			\end{equation}
			Таким образом, коэффициенты $\mathcal{B}_{j,i}^{n+1,q}$ могут быть легко найдены обращением матриц с диагоналями $\alpha_{j,i},\beta_{j,i},\gamma_{j,i}$ методом прогонки \cite{abramov}.
		\subsection{Граничные условия}
			\par Одной из особенностей МПУ является то, что их решение всегда рассмат­ривается в неограниченной области, в отличие от решения «обычных», или «вер­тикальных», параболических уравнений в подводной акустике, которое вычис­ляется в нескольких слоях, имеющих в качестве верхней границы поверхность океана и, как минимум в теории, некоторую границу на дне моря. Таким образом, искусственное ограничение вычислительной области является обязательным при численном решении МПУ. В рамках данной работы рассматривается два способа выполнения такого ограничения. В первом случае вычислительная области расширяется с обеих сторон с целью поглощения исходящих волн. Во втором, решение внутри области сопоставляется с решением, содержащим исходящие волны за её пределами, с использованием специальных искусственных граничных условий.
			\subsubsection{Согласованные поглощающие слои\label{sec::PML}}
				\par Впервые метод согласованных поглощающих слоёв (perfectly matching layers, PML) для граничных условий был показан Беранже для уравнений Максвелла \cite{berenger}, а позднее Леви \cite{levy}, Лу и Чжу \cite{lu} исследовали применимость этого метода для решения параболических уравнений. В основе PML лежит расширение вычислительной области с целью плавного поглощения волн исходящих из неё. Пусть требуется найти решение уравнения \eqref{eq::PDMPE} в области $\Omega=\left[0,x_1\right]\times\left[y_0,y_1\right]$. Сформируем новую область $\overline{\Omega}=\left[0,x_1\right]\times\left[y_0-\varepsilon,y_1+\varepsilon\right]$, расширив $\Omega$ на $\varepsilon$ с обоих сторон вдоль оси $y$. Заменим оператор $L_j$ в уравнении \eqref{eq::PDMPE} оператором $L_j^{PML}$, определяемым как
				\begin{equation}\label{eq::LPML}
					k_{j,0}^2L_j^{PML}=\frac{1}{1+i\beta\pa{y}}\frac{\partial}{\partial y}\frac{1}{1+i\beta\pa{y}}\frac{\partial}{\partial y}+k_j^2+k_{j,0}^2\,,
				\end{equation}
				где $\beta\pa{y}$ -- некоторая гладкая функция, монотонно убывающая на интервале\\ $\left[y_0-\varepsilon,y_0\right)$, возрастающая на $\left(y_1,y_1+\varepsilon\right]$, и $\beta\pa{y}=0$ при $y\in\left[y_0,y_1\right]$. Таким образом, оператор $L_j^{PML}$ совпадает с оператором $L_j$ внутри области $\Omega$. На границах $y_0-\varepsilon$, $y_1+\varepsilon$ устанавливаются стандартные однородные условия Дирихле
				\begin{equation}
					\mathcal{A}_j\bigr|_{y=y_0-\varepsilon}=\mathcal{A}_j\bigr|_{y=y_1+\varepsilon}=0\,.
				\end{equation}
				Таким образом, применимость PML граничных условий зависит от значения параметра $\varepsilon$, который должен быть существенно большим, чтобы погладить исходящие волны, и функции $\beta\pa{y}$, которая в свою очередь должна существенно гладко достигать своего максимального значения при погружении в PML, так, при слишком быстром возрастании исходящие волны будут отражаться внутри поглощающего слоя.
				\par В рамках данной работы была использована следующая функция $\beta\pa{y}$
				\begin{equation}\label{eq::betay}
					\beta\pa{y}=\beta_0\pa{\frac{\left|y-y_b\right|}{\varepsilon}}^3=\beta_0\zeta^3\equiv\beta\pa{\zeta}\,,\quad\zeta\in\left[0, 1\right]\,,
				\end{equation}
				где $y_b$ -- соответствующая граница области $\Omega$, $\beta_0\in\mathbb{R}_{+}$ -- параметр масштаба.
				\paragraph{Дискретизация оператора $L_j^{PML}$}
					\par Для дискретизации оператора $L_j^{PML}$ на равномерной сетке $y_q=y_0-\varepsilon+q\delta$ используется стандартная конечно-разностная схема для первой производной с половинным шагом
					\begin{equation}\label{eq::D1}
						D^1_{\nicefrac{\delta}{2}}F^q=\frac{F^{q+\nicefrac{1}{2}}-F^{q-\nicefrac{1}{2}}}{\delta}\,.
					\end{equation}
					Таким образом, выражение $\eqref{eq::BEQ}$ преобразуется к виду 
					\begin{equation}
						\pa{1+b_{l,m}^i\prescript{q}{\delta}{L_j^{PML}}}\mathcal{B}_{j,i}^{n+1,q}=\mathcal{A}_j^{n,q}\,,\quad i=\overline{1,p}\,,
					\end{equation}
					где
					\begin{equation}\label{eq::LQDPML}
						k_{j,0}^2\prescript{q}{\delta}{L_j^{PML}}=\frac{1}{1+i\beta\pa{y_q}}D_{\nicefrac{q}{2}}^1\pa{\frac{1}{1+i\beta\pa{y_q}}D_{\nicefrac{q}{2}}^1}+k_j^2-k_{j,0}^2\,.
					\end{equation}
					Используя равенства \eqref{eq::D1} и \eqref{eq::LQDPML}, получим
					\begin{multline}
						\underset{\tilde{\alpha}_{j,i}^q}{\underbrace{\frac{b_{l,m}^i\mu_q\mu_{q-\nicefrac{1}{2}}}{k_0^2\delta^2}}}\mathcal{B}_{j,i}^{n+1,q-1}+\underset{\tilde{\beta}_{j,i}^q}{\underbrace{\pa{1+\frac{b_{l,m}^i}{k_{j,0}^2}\pa{k_j^2-k_{j,0}^2-\frac{\mu_q}{\delta^2}\pa{\mu_{q-\nicefrac{1}{2}}+\mu_{q+\nicefrac{1}{2}}}}}}}\mathcal{B}_{j,i}^{n+1,q}+\\\underset{\tilde{\gamma}_{j,i}^q}{\underbrace{\frac{b_{l,m}^i\mu_q\mu_{q+\nicefrac{1}{2}}}{k_0^2\delta^2}}}\mathcal{B}_{j,i}^{n+1,q+1}=\mathcal{A}_j^{n,q}\,,
					\end{multline}
					где $\mu_q=\nicefrac{1}{\pa{1+i\beta\pa{y_q}}}$. Так как оператор $L_j^{PML}$ совпадает с оператором $L_j$ внутри области $\Omega$, значения вспомогательных функций $\mathcal{B}_{j,i}^{n,q}$ в расширенной области $\overline{\Omega}$ могут быть найдены обращением матриц методом прогонки \cite{abramov} с диагоналями $\tilde{\alpha}_{j,i}^q,\tilde{\beta}_{j,i}^q,\tilde{\gamma}_{j,i}^q$ на интервалах $\left[y_0-\varepsilon,y_0\right)$, $\left(y_1,y_1+\varepsilon\right]$ и $\alpha_{j,i},\beta_{j,i},\gamma_{j,i}$ на отрезке $\left[y_0,y_1\right]$.
			\subsubsection{Граничные условия прозрачности}
				\par Граничные условия, позволяющие предотвратить отражение от произвольной границы называются прозрачными граничными условиями. Для узкоугольного параболического уравнения они были впервые получены в непрерывной форме Баскаковым и Поповым \cite{baskakov1991} и Маркусом \cite{marcus1991}. В дальнейшем Попвым также были получены условия для широкоугольного параболического уравнения в самой простой форме. Полностью дискретные условия для узкоугольного параболического уравнения пригодные для использования в численном моделировании были получены Арнольдом и Эрхардтом \cite{ehrhardt2002,xavier2008}, и в дальнейшем были расширены на случай широкоугольных параболических уравнений \cite{arnold1998}.
				\par Рассмотрим уравнения \eqref{eq::BEQ} и, учитывая, что $k_{j,0}^2L_j^\delta=D_\delta^2+k_j^2-k_{j,0}^2$, получим
				\begin{equation}\label{eq::TBC_start}
					k_{j,0}^2\mathcal{B}_{j,i}^{n+1,q}+b_{l,m}^iD_\delta^2\mathcal{B}_{j,i}^{n+1,q}+b_{l,m}^i\pa{k_j^2-k_{j,0}^2}\mathcal{B}_{j,i}^{n+1,q}=k_{j,0}^2\mathcal{A}_j^{n,q}\,.
				\end{equation}
				Решение такой системы может быть найдено с использованием $\mathcal{Z}$-преобра\-зования по эволюционной координате $x$
				\begin{equation}
					\mathcal{Z}\left\{\mathcal{A}_j^{n,q}\right\}=\hat{\mathcal{A}}_j^q\pa{\zeta}\coloneq\sum\limits_{n=0}^\infty\zeta^{-n}\mathcal{A}_j^{n,q}\,,\quad\zeta\in\mathbb{C}\,,\quad\left|\zeta\right|>R_{\hat{\mathcal{A}}_j^q}\,,
				\end{equation}
				где $R_{\hat{\mathcal{A}}_j^q}$ -- радиус сходимости ряда Лорана. Применив это преобразование к \eqref{eq::TBC_start} и дополнительно потребовав $\mathcal{A}_j^{n,q}=\mathcal{B}_j^{n,q}=0,\forall n<0$, получим
				\begin{equation}
					-\zeta b_{l,m}^iD_\delta^2\hat{\mathcal{B}}_{j,i}^q=\zeta k_{j,0}^2\hat{\mathcal{B}}_{j,i}^q+\zeta b_{l,m}^i\pa{k_j^2-k_{j,0}^2}\hat{\mathcal{B}}_{j,i}^q-k_{j,0}^2\hat{\mathcal{A}}_j^q\,.
				\end{equation}
				Обозначим $\hat{\bm{\Psi}}_j^q=\pa{\hat{\mathcal{A}}_j^q,\hat{\mathcal{B}}_{j,1}^q,\dots,\hat{\mathcal{B}}_{j,p}^q}^T\in\mathbb{C}^{p+1}$, добавим к системе $Z$-преобра\-зование уравнения \eqref{eq::BtoA}
				\begin{equation}
					\zeta\hat{\mathcal{A}}_j^q=a_{l,m}^0\hat{\mathcal{A}}_j^q+\zeta\sum\limits_{i=1}^pa_{l,m}^i\hat{\mathcal{B}}_{j,i}^q\,,
				\end{equation}
				и запишем полученную систему в матричном виде
				\begin{equation}\label{eq::TBCmatrixEQ}
					\mathbf{X}_jD_\delta^2\hat{\bm{\Psi}}_j^q=\mathbf{Y}_j\hat{\bm{\Psi}}_j^q\,,
				\end{equation}
				где $\mathbf{X}_j,\mathbf{Y}_j$ -- комплекснозначные $\pa{p+1}\times\pa{p+1}$ матрицы вида
				\begin{equation}
					\mathbf{X}_j\coloneq\begin{pmatrix}
						0 & -\zeta b_{l,m}^1&&\\
						\vdots& & \ddots&&\\
						0&&&-\zeta b_{l,m}^p\\
						\zeta&0&\dots&0
					\end{pmatrix}\,,
				\end{equation}
				и
				\begin{equation}
					\mathbf{Y}_j\coloneq\begin{pmatrix}
						-k_{j,0}^2&\zeta k_{j,0}^2+\zeta b_{l,m}^1\pa{k_j^2-k_{j,0}^2}&&\\
						\vdots&&\ddots&\\
						-k_{j,0}^2&&&\zeta k_{j,0}^2+\zeta b_{l,m}^p\pa{k_j^2-k_{j,0}^2}\\
						a_{l,m}^0&\zeta a_{l,m}^1&\dots&\zeta a_{l,m}^p
					\end{pmatrix}\,.
				\end{equation}
				Введём новую переменную $\hat{\bm{\xi}}_j^q\coloneq D_q^-\hat{\bm{\Psi}}_j^q$ и запишем систему \eqref{eq::TBCmatrixEQ} в виде системы $2\pa{p+1}$ дифференциальных уравнений первого порядка
				\begin{equation}
					\underbrace{\begin{pmatrix}
						\mathbf{0} & \mathbf{X}_j\\
						\mathbf{I} & -\mathbf{I}\delta
					\end{pmatrix}}_{\mathbf{A}_j} D_\delta^+
					\begin{pmatrix}
						\hat{\bm{\Psi}}_j^q\\
						\hat{\bm{\xi}}_j^q
					\end{pmatrix}=
					\underbrace{\begin{pmatrix}
						\mathbf{Y}_j & \mathbf{0}\\
						\mathbf{0} & \mathbf{I}
					\end{pmatrix}}_{\mathbf{B}_j}
					\begin{pmatrix}
						\hat{\bm{\Psi}}_j^q\\
						\hat{\bm{\xi}}_j^q
					\end{pmatrix}\,,
				\end{equation}
				где $D_q^-\hat{\bm{\Psi}}_j^q=\frac{\hat{\bm{\Psi}}_j^q-\hat{\bm{\Psi}}_j^{q-1}}{\delta}$ и $D_q^+\hat{\bm{\Psi}}=\frac{\hat{\bm{\Psi}}_j^{q+1}-\hat{\bm{\Psi}}_j^q}{\delta}$ -- конечные разности назад и вперёд соответственно. Решение этой системы может быть записано в следующем виде
				\begin{equation}
					\begin{pmatrix}
						\hat{\bm{\Psi}}_j^{q+1}\\
						\hat{\bm{\xi}}_j^{q+1}
					\end{pmatrix}=
					\pa{\mathbf{A}_j^{-1}\mathbf{B}_j+\mathbf{I}}
					\begin{pmatrix}
						\hat{\bm{\Psi}}_j^q\\
						\hat{\bm{\xi}}_j^q
					\end{pmatrix}\,.
				\end{equation}
				Запишем Жорданову форму матрицы $\mathbf{A}_j^{-1}\mathbf{B}_j+\mathbf{I}$
				\begin{equation}
					\mathbf{J}_j=\begin{pmatrix}
						\mathbf{J}_j^1 & \mathbf{0}\\
						\mathbf{0} & \mathbf{J}_j^2
					\end{pmatrix}=
					\mathbf{P}_j\pa{\mathbf{A}_j^{-1}\mathbf{B}_j+\mathbf{I}}\mathbf{P}_j^{-1}\,,
				\end{equation}
				где $\mathbf{J}_j^1,\mathbf{J}_j^2\in\mathbb{C}^{\pa{p+1}\times\pa{p+1}}$ -- матрицы, содержащие Жордановы блоки, соответствующие решениям, затухающим и возрастающим при $q\rightarrow\infty$ соответственно,
				и $\mathbf{P}_j^{-1}$ -- матрица левых собственных значений вида
				\begin{equation}
					\mathbf{P}_j^{-1}=\begin{pmatrix}
						\mathbf{P}_j^1 & \mathbf{P}_j^2\\
						\mathbf{P}_j^3 & \mathbf{P}_j^4
					\end{pmatrix}\,.
				\end{equation}
				Тогда справедливо разложение решения по базису $\mathbf{P}_j^{-1}$
				\begin{multline}
					\mathbf{P}_j^{-1}\begin{pmatrix}
						\hat{\bm{\Psi}}_j^{q+1}\\
						\hat{\bm{\xi}}_j^{q+1}
					\end{pmatrix}=
					\mathbf{P}_j^{-1}\pa{\mathbf{A}_j^{-1}\mathbf{B}_j+\mathbf{I}}\begin{pmatrix}
						\hat{\bm{\Psi}}_j^q\\
						\hat{\bm{\xi}}_j^q
					\end{pmatrix}=\\
					\mathbf{P}_j^{-1}\mathbf{P}\begin{pmatrix}
						\mathbf{J}_j^1 & \mathbf{0}\\
						\mathbf{0} & \mathbf{J}_j^2
					\end{pmatrix}
					\begin{pmatrix}
						\mathbf{P}_j^1 & \mathbf{P}_j^2\\
						\mathbf{P}_j^3 & \mathbf{P}_j^4
					\end{pmatrix}
					\begin{pmatrix}
						\hat{\bm{\Psi}}_j^q\\
						\hat{\bm{\xi}}_j^q
					\end{pmatrix}=
					\begin{pmatrix}
						\mathbf{J}_j^1 & \mathbf{0}\\
						\mathbf{0} & \mathbf{J}_j^2
					\end{pmatrix}
					\begin{pmatrix}
						\mathbf{P}_j^1\hat{\bm{\Psi}}_j^q+\mathbf{P}_j^2\hat{\bm{\xi}}_j^q\\
						\mathbf{P}_j^3\hat{\bm{\Psi}}_j^q+\mathbf{P}_j^4\hat{\bm{\xi}}_j^q
					\end{pmatrix}\,.
				\end{multline}
				Требуя равенства нулю выражения, соответствующего решениям, возрастающим при $q\rightarrow\infty$, получим преобразованные прозрачные граничные условия на правой границе области при $q=Q$
				\begin{equation}
					\mathbf{P}_j^3\hat{\bm{\Psi}}_j^Q+\mathbf{P}_j^4\hat{\bm{\xi}}_j^Q=0\,.
				\end{equation}
				Таким образом, $\mathcal{Z}$-преобразованные правые прозрачные граничные условия с невырожденной матрицей $\mathbf{P}_j^4$ могут быть записаны в конечно-разностной форме
				\begin{equation}
					D_\delta^-\hat{\bm{\Psi}}_j^Q=\hat{\mathbf{D}}_j\hat{\bm{\Psi}}_j^Q\,,
				\end{equation}
				где $\hat{\mathbf{D}}_j=-\pa{\mathbf{P}_j^4}^{-1}\mathbf{P}_j^3$. Применяя обратное $\mathcal{Z}$-преобразование получим дискретные правые прозрачные граничные условия
				\begin{equation}
					\bm{\Psi}_j^{n+1,Q}-\bm{\Psi}_j^{n+1,Q-1}=\sum\limits_{l=1}^{n+1}\mathbf{D}_j^{n+1-l}\bm{\Psi}_j^{l,Q}\,,
				\end{equation}
				для вектора $\bm{\Psi}_j^{n,q}=\pa{\mathcal{A}_j^{n,q},\mathcal{B}_{j,1}^{n,q},\dots,\mathcal{B}_{j,1}^{n,q}}^T\in\mathbb{C}^{p+1}$, при этом коэффициенты $\mathbf{D}_j^n$ разложения могут быть получены из интегральной формулы Коши
				\begin{equation}
					\mathbf{D}_j^n=\mathcal{Z}^{-1}\left\{\hat{\mathbf{D}}_j\pa{\zeta}\right\}=\frac{\tau^n}{2\pi}\int\limits_0^{2\pi}\hat{\mathbf{D}}_j\pa{\tau e^{i\phi}}e^{in\phi}\dd\phi\,,\quad n\in\mathbb{Z}_0\,,\quad\tau > 0\,.
				\end{equation}
				\par Граничные условия на левой границе области при $q=0$ могут быть получены аналогично из требования равенства нулю  выражения, соответствующего решениям, затухающим при $q\rightarrow\infty$, то есть возрастающим при $q\rightarrow-\infty$
				\begin{equation}
					\mathbf{P}_j^1\hat{\bm{\Psi}}_j^0+\mathbf{P}_j^2\hat{\bm{\xi}}_j^0=0\,.
				\end{equation}
			\subsection{Лучевая теория для уравнения горизонтальной рефракции}
				\par Предполагая, что волновые числа $k_j\pa{x,y}$ являются медленно изменяющейся функцией, решение уравнения \eqref{eq::HRE} с использованием лучевой теории распространения звука может быть выражено в виде
				\begin{equation}
					A_j\pa{x,y}=M_j\pa{x,y}e^{ik_{j,0}S_j\pa{x,y}}+o\pa{\nicefrac{1}{k_{j,0}}}\,,
				\end{equation}
				где функция $S_j\pa{x,y}$ называется эйконалом и может быть найдена из уравнения Гамильтона-Якоби
				\begin{equation}\label{eq::hamilton_jacoby}
					\pa{\frac{\partial S_j\pa{x,y}}{\partial x}}^2+\pa{\frac{\partial S_j\pa{x,y}}{\partial y}}^2=n_j\pa{x,y}\,,
				\end{equation}
				где $n_j\pa{x,y}\equiv \nicefrac{k_j\pa{x,y}}{k_{j,0}}$ --- индекс горизонтальной рефракции \cite{burridge}. Амплитуда $M_j\pa{x,y}$ может быть получена из уравнения переноса вида
				\begin{multline}\label{eq::transfer}
					2\pa{\frac{\partial S_j\pa{x,y}}{\partial x}\frac{\partial M_j\pa{x,y}}{\partial x}+\frac{\partial S_j\pa{x,y}}{\partial y}\frac{\partial M_j\pa{x,y}}{\partial y}}+\\\pa{\frac{\partial^2S_j\pa{x,y}}{\partial x^2}+\frac{\partial^2S_j\pa{x,y}}{\partial y^2}}M_j\pa{x,y}=0\,.
				\end{multline}
				Решение этих уравнений связано с решением системы Гамильтона
				\begin{equation}\label{eq:hamilton_system}
					\begin{aligned}
						\frac{dx_j\pa{l}}{dl}&=\frac{\xi_j\pa{l}}{n_j\pa{x_j,y_j}}\,,\qquad&\frac{d\xi_j\pa{l}}{dl}&=\frac{\partial n_j\pa{x_j,y_j}}{\partial x_j}\,,\\
						\frac{dy_j\pa{l}}{dl}&=\frac{\eta_j\pa{l}}{n_j\pa{x_j,y_j}}\,,\qquad&\frac{d\eta_j\pa{l}}{dl}&=\frac{\partial n_j\pa{x_j,y_j}}{\partial y_j}\,,\\
					\end{aligned}
				\end{equation}
				где $l$ является натуральным параметром, обозначающим длину кривой вдоль траектории распространения луча, а $\xi,\eta$ сопряжённые переменные к $x,y$ --- момент. Проекции решений этой системы из фазового пространства $\pa{x,y,\xi,\eta}$ на координатное $\pa{x,y}$ называются горизонтальными лучами, соответствующими вертикальным модам в океаническом волноводе.  
			\subsubsection{Трассировка лучей, соответствующих вертикальным модам\label{sec::horizontal_rays}}
				\paragraph{Математическая постановка задачи}
					\par Задача трассировки лучей состоит в том, чтобы в области\\ $\left\{\pa{x,y,z_s}0\leqslant x\leqslant x_1,y_0\leqslant y\leqslant y_1\right\}$ вычислить координаты распространения лучей, соответствующих вертикальным модам, из источника, имеющего координаты $\pa{0,y_s,z_s}$, под углами распространения $\alpha_0\leqslant\alpha\leqslant\alpha_1$ и значениях натурального параметра кривой $l_0\leqslant l\leqslant l_1$, задаваемые Гамильтовой системой задач Коши
					\begin{equation}
						\begin{aligned}
							\begin{dcases}
								\frac{dx_j\pa{l}}{dl}=\frac{\xi_j\pa{l}}{n_j\pa{x_j,y_j}}\,,\\
								x_j\pa{0}=0\,,
							\end{dcases}&\qquad
							\begin{dcases}
								\frac{d\xi_j\pa{l}}{dl}=\frac{\partial n_j\pa{x_j,y_j}}{\partial x_j}\,,\\
								\xi_j\pa{0}=\cos\alpha\,,
							\end{dcases}\\
							&&\\
							\begin{dcases}
								\frac{dy_j\pa{l}}{dl}=\frac{\eta_j\pa{l}}{n_j\pa{x_j,y_j}}\,,\\
								y_j\pa{0}=y_s\,,
							\end{dcases}&\qquad
							\begin{dcases}
								\frac{d\eta_j\pa{l}}{dl}=\frac{\partial n_j\pa{x_j,y_j}}{\partial y_j}\,,\\
								\eta_j\pa{0}=\sin\alpha\,.
							\end{dcases}
						\end{aligned}
					\end{equation}
				\paragraph{Явные методы Рунге-Кутты}
					\par Явные методы Рунге-Кутты являются семейством численных методов для решения систем обыкновенных дифференциальных уравнений вида
					\begin{equation}
						\begin{dcases}
							\boldsymbol{y}^{\prime}_x\pa{x}=\boldsymbol{f}\pa{x,\boldsymbol{y}\pa{x}}\,,\\
							\boldsymbol{y}\pa{x_0}=\boldsymbol{y}_0\,,
						\end{dcases}
					\end{equation}
					где $\boldsymbol{y}:\mathbb{R}\mapsto\mathbb{R}^n$ --- искомая функция, $\boldsymbol{f}:\mathbb{R}^{n+1}\mapsto\mathbb{R}^n$ --- функция зависимости, $\boldsymbol{y}_0\in\mathbb{R}^n$ --- начальное значение функции, $x,x_0\in\mathbb{R}$ --- аргумент и его начальное значение.
					\par Пусть задана некоторая дискретная сетка\\ $\Omega=\left\{\pa{x_i,\boldsymbol{y}_i}\bigr|x_i=x_{i-1}+h_i,\boldsymbol{y}_i\approx\boldsymbol{y}\pa{x_i},x_i,h_i\in\mathbb{R},\boldsymbol{y}_i\in\mathbb{R}^n\right\}$, тогда явный $s$-ша\-говый метод Рунге-Кутты может быть записан в виде
					\begin{equation}
						\boldsymbol{y}_{i+1}=\boldsymbol{y}_i+h_i\sum\limits_{j=1}^{s}b_j\boldsymbol{k}_j
					\end{equation}
					где $\boldsymbol{k}$ значения функции $\boldsymbol{f}$, вычисленные в специальных промежуточных точках интервала
					\begin{equation}
						\begin{aligned}
							\boldsymbol{k_1}&=\boldsymbol{f}\pa{x_i,\boldsymbol{y}_i}\,,\\
							\boldsymbol{k_2}&=\boldsymbol{f}\pa{x_i+c_2h_i,\boldsymbol{y}_i+a_{2,1}h_i\boldsymbol{k}_1}\,,\\
							\dots&\\
							\boldsymbol{k_s}&=\boldsymbol{f}\pa{x_i+c_sh_i,\boldsymbol{y}_i+a_{s,1}h_i\boldsymbol{k}_1+a_{s,2}h_i\boldsymbol{k}_2+\dots+a_{s,s-1}h_i\boldsymbol{k}_{s-1}}
						\end{aligned}
					\end{equation}
					или в общем виде 
					\begin{equation}
						\boldsymbol{k}_j=\boldsymbol{f}\pa{x_i+c_jh_i,\boldsymbol{y}_i+h_i\sum\limits_{t=1}^{j-1}a_{j,t}\boldsymbol{k}_t}\,,\qquad j=\overline{1,s}\,.
					\end{equation}
					Конкретный метод порядка $p$ задаётся коэффициентами $a_{j,t},b_j,c_j\in\mathbb{R}$, часто записываемыми в виде таблицы
					\begin{equation}
						\begin{array}{c|ccccc}
							0 & \multicolumn{5}{c}{}\\
							c_2 & a_{2,1} & \multicolumn{4}{c}{}\\
							c_3 & a_{3,1} & a_{3,2} & \multicolumn{3}{c}{}\\
							\vdots & \vdots & \vdots & \ddots & \multicolumn{2}{c}{}\\
							c_s & a_{s,1} & a_{s,2} & \dots & a_{s,s-1} & \\
							\hline
							& b_1 & b_2 & \dots & b_{s-1} & b_s
						\end{array}
					\end{equation}
					и удовлетворяют условиям
					\begin{gather}
						\sum\limits_{t=1}^{j-1}a_{j,t}=c_j\,,\qquad j=\overline{1,s}\,,\\
						\boldsymbol{y}_i-\boldsymbol{y}\pa{x_i}=O\pa{h_i^{p+1}}\,.
					\end{gather}
				\paragraph{Плотная выдача}
					\par При уменьшении шага сетки использование методов Рунге-Кутты становится менее эффективными или невозможным из-за необходимости вычислять промежуточные значения функции $\boldsymbol{f}$. Одним из вариантов ускорения процесса вычислений является так называемая плотная выдача, позволяющая вычислять значения $\boldsymbol{y}\pa{x_i+\theta h_i}$ на всём отрезке $0\leqslant\theta\leqslant1$, при этом в худшем случае используя лишь незначительное по сравнению с основным методом количество вычислений функции $\boldsymbol{f}$. Метод $s^{\star}$-шаговой плотной выдачи может быть в общем виде записан как 
					\begin{equation}
						\boldsymbol{u}_i\pa{\theta}=\boldsymbol{y}_i+h_i\sum\limits_{j=1}^{s^{\star}}q_j\pa{\theta}\boldsymbol{k}_j\,,
					\end{equation}
					где 
					\begin{equation}
						\boldsymbol{k}_j=\boldsymbol{f}\pa{x_i+c_jh_i,y_i+h_i\sum\limits_{t=1}^{j-1}a_{j,t}\boldsymbol{k}_t}\,,\qquad j=\overline{1,s^{\star}}\,,
					\end{equation}
					при этом $s^{\star}\geqslant s$, а зачастую $s^{\star}\geqslant s+1$, так как значение $\boldsymbol{k}_{s+1}=\boldsymbol{f}\pa{\boldsymbol{x_i}+h_i,\boldsymbol{y}_{i+1}}$ может быть получено с вычислением следующего шага метода при $a_{s+1,t}=b_{t}\,,t=\overline{1,s}$. Порядок $p^{\star}$ плотной выдачи определяется количеством шагов $s^{\star}$ и видом полиномов $q_j\pa{\theta}$ и задаётся условием
					\begin{equation}
						\boldsymbol{u}_i\pa{\theta}-\boldsymbol{y}\pa{x_i+\theta h_i}=O\pa{h_i^{p^{\star}-1}}\,.
					\end{equation}
					Было показано, что с использованием следующего представления полиномов $q_j\pa{\theta}$
					\begin{equation}
						q_j\pa{\theta}=\sum\limits_{l=1}^{p^{\star}}q_{j,l}\theta^l
					\end{equation}
					порядок $p^{\star}$ может быть достигнут увеличением количества шагов $s^{\star}$ \cite{dense}. Таким образом непрерывный явный метод Рунге-Кутты может быть задан двумя таблицами
					\begin{equation}
						\begin{array}{c|ccccccccc}
							0 & \multicolumn{9}{c}{}\\
							c_2 & a_{2,1} & \multicolumn{8}{c}{}\\
							c_3 & a_{3,1} & a_{3,2} & \multicolumn{7}{c}{}\\
							\vdots & \vdots & \vdots & \ddots & \multicolumn{6}{c}{}\\
							c_s & a_{s,1} & a_{s,2} & \dots & a_{s,s-1} & \multicolumn{5}{c}{}\\
							\hline
							1 & b_1 & b_2 & \dots & b_{s-1} & b_s & \multicolumn{4}{c}{}\\
							\hline
							c_{s+2} & a_{s+2,1} & a_{s+3,2} & \dots & a_{s-1} & a_{s} & a_{s + 1} & \multicolumn{3}{c}{}\\
							\vdots & \vdots & \vdots & \vdots & \vdots & \vdots & \vdots & \ddots & \multicolumn{2}{c}{}\\
							c_{s^{\star}} & a_{s^{\star},1} & a_{s^{\star},2} & \dots & \dots & \dots & \dots & \dots & a_{s^{\star},s^{\star}-1}
						\end{array}\qquad\qquad
						\begin{array}{cccc}
							q_{1,1} & q_{1,2} & \dots & q_{1,p^{\star}}\\
							\vdots & \vdots & \vdots & \vdots\\
							q_{s^{\star},1} & q_{s^{\star},2} & \dots & q_{s^{\star},p^{\star}}\\
						\end{array}
					\end{equation}
				\paragraph{Автоматический контроль шага сетки}
					\par Методы Рунге-Кутты также позволяют производить автоматический контроль шага с целью оптимизации вычислений и гарантированного поддержания заданной ошибки \cite{dense}. Для этого задаются дополнительные коэффициенты $\hat{b}_j, j=\overline{1,j}$, соответствующие тем же коэффициентам $a_{j,t}$ и $c_j$, обеспечивающие точность $\hat{p}$. На каждом шаге помимо $\boldsymbol{y}_{i+1}$ также вычисляется $\boldsymbol{\hat{y}}_{i+1}$ и производится оценка ошибки
					\begin{equation}
						err=\sqrt{\frac{1}{n}\sum\limits_{l=1}^n\pa{\frac{y_{i+1,l}-\hat{y}_{y+1,l}}{Atol+\max\pa{\left|y_{i,l}\right|, \left|y_{y+1,l}\right|}\cdot Rtol}}^2}
					\end{equation}
					с последующим обновлением шага сетки
					\begin{equation}
						\hat{h}=h_i\cdot\pa{\nicefrac{1}{err}}^{\nicefrac{1}{q+1}}
					\end{equation}
					где $q=\min\pa{p,\hat{p}}$. Для увеличения вероятности принятия нового шага на следующем этапе вычислений вводится коэффициент $fac$, который часто выбирается как $fac=0.8\,,0.9\,,\pa{0.25}^{\nicefrac{1}{q+1}}\,,\pa{0.38}^{\nicefrac{1}{q+1}}$. Также для увеличения стабильности вводятся коэффициенты $facmax$ и $facmin$, ограничивающие скорость изменения размера шага. Таким образом,
					\begin{equation}
						\hat{h}=h_i\cdot\min\pa{facmax,\max\pa{facmin,fac\cdot\pa{\nicefrac{1}{err}}^{\nicefrac{1}{q+1}}}},.
					\end{equation}
					Затем, если $err\leqslant1$ новый шаг сетки принимается и продолжается вычисление решения уравнения с шагом $h_{i+1}=\hat{h}$. Иначе новый шаг отвергается и вычисления производятся заново при $h_i=\hat{h}$.
		\subsection{Начальные условия}
			\subsubsection{Начальные условия Гаусса и Грина}
				\par От выбора начальных условий зависит устойчивость получаемого численного решения. Для параболических уравнений наиболее часто используется начальное условие Гаусса \cite{jensen}
				\begin{equation}\label{eq::gauss}
					\mathcal{A}_j\pa{0,y}=\frac{\varphi_j\pa{z_s}}{2\sqrt{\pi}}e^{-k_{j,0}^2\pa{y-y_s}}\,.
				\end{equation}
				Однако такое условие создаёт большой численный шум при использовании даже небольшого порядка аппроксимации квадратного корня. Также может быть использовано начальное условие Грина
				\begin{equation}\label{eq::greene}
					\mathcal{A}_j\pa{0,y}=\frac{\varphi_j\pa{z_s}}{2\sqrt{\pi}}\pa{1.4467-0.8402k_{j,0}^2\pa{y-y_s}^2}e^{-\frac{k_{j,0}^2\pa{y-y_s}^2}{1.5256}}\,,
				\end{equation}
				которое обеспечивают большую, но всё ещё не идеальную стабильность.
			\subsubsection{Лучевые начальные условия}\label{sec::ray_starters}
				\par Для использования высоких порядков аппроксимации Паде необходимо начальное условие, учитывающее широкоугольные особенности решаемого уравнения. Такое условие может быть получено с использованием лучевой теории распространения звука. Предположим, что при $0\leqslant x\leqslant x_0$, где $x_0$ сравнимо с длинной волны, свойства среды не зависят от $x$, то есть $k_j=k_j\pa{y}$. Тогда, решение \eqref{eq::PDMPE} может быть записано в виде
				\begin{equation}
					\mathcal{A}_j\pa{x,y}=M_j\pa{x,y}e^{ik_{0,j}S_j\pa{x,y}}+o\pa{\nicefrac{1}{k_{0,j}}}\,,
				\end{equation}
				где $M_j\pa{x,y}$ --- амплитуда нулевого порядка, удовлетворяющая уравнению переноса \eqref{eq::transfer},
				$S_j\pa{x,y}$ --- функция эйконала, удовлетворяющая уравнению Гамильтона-Якоби \eqref{eq::hamilton_jacoby}. 
				Оба эти уравнения могут быть получены из решения системы Гамильтона \eqref{eq:hamilton_system} вдоль кривой распространения звукового луча в виде
				\begin{align}
					S_j\pa{l}&=\int\limits_0^ln_j\pa{l}dl\,,\\
					M_j\pa{l}&=\frac{M_{j,0}}{n_j\pa{l}}\sqrt{\frac{\cos\alpha}{\nicefrac{\partial y\pa{l,\alpha}}{\partial\alpha}}}\,,
				\end{align}
				где $M_{j,0}=\nicefrac{e^{\nicefrac{i\pi}{4}}}{\sqrt{8\pi k_{j,0}}}$ --- амплитуда на расстоянии 1 м. от источника, $n_j\pa{l}=\nicefrac{k_j\pa{l}}{k_{j,0}}$ -- показатель горизонтального преломления.
				\par Так как значения функций $S_j\pa{x,y}$ и $M_j\pa{x,y}$ вычисляются лишь на небольшом расстоянии $x_0$, в большинстве случаев будет достаточно начального условия рассчитанного для однородной среды при $k_j=\pa{x,y}=k_{j,0}$. В таком случае получим
				\begin{equation}\label{eq::ray_simple}
					\begin{aligned}
						S_j\pa{y}&=r\pa{y}\,,\\
						M_j\pa{y}&=\frac{M_{j,0}}{\sqrt{r\pa{y}}}\,,\\
						r\pa{y}&=\sqrt{x_0^2+y^2}\,.
					\end{aligned}
				\end{equation}
		\subsection{Расчёт временных рядов в точках приёма при распространении импульсных акустических сигналов\label{sec::impulse}}
			\subsubsection{Математическая постановка задачи}
				\par Задача состоит в вычислении в точках приёма $R=\left\{\pa{x,y,z}\in\Omega\right\}$ области $\Omega=\left\{\pa{x,y,z}\bigr|x_0\leqslant x\leqslant x_1,y_0\leqslant y\leqslant y_1,0\leqslant z\leqslant z_b\right\}$ временного ряда импульса сигнала в источнике $S=\pa{x_0, y_s, z_s}$, задаваемого функцией $g\pa{t}\,,t_0\leqslant t\leqslant t_1$. Импульс $I_r\pa{t}$ в приёмнике $r$ в спектральной области Фурье определяется следующей функцией (здесь и далее оператор $\hat{\pa{\cdot}}$ означает функцию в спектральной области)
				\begin{equation}
					\hat{I}_r\pa{\xi}=\overline{\hat{P}\pa{x_r,y_r,z_r, \xi}\cdot e^{-i\tau\omega\pa{\xi}}}\,,
				\end{equation}
				где $\omega\pa{\xi}=2\pi f\pa{\xi}$ -- циклическая частота источника, $f\pa{\xi}$ -- частота источника, $\tau$ -- время движения звука из источника в приёмник, $\overline{\pa{\cdot}}$ -- оператор комплексного сопряжения, $\hat{P}$ -- функция сигнала в приёмнике
				\begin{equation}\label{eq::receiver_signal}
					\hat{P}\pa{x,y,z,\xi}=p\pa{x,y,z,f\pa{\xi}}\cdot\overline{\hat{g}}\pa{\xi}\,,
				\end{equation}
				здесь $p\pa{x,y,z,f\pa{\xi}}$ -- звуковое поле источника, вычисленное для частоты $f\pa{\xi}$. Значение $\hat{g}\pa{\xi}$ также может быть оценено как
				\begin{equation}\label{eq::ref}
					\hat{g}\pa{\xi}=\frac{\hat{I}_{r_0}\pa{\xi}}{p\pa{x,y,z,f\pa{\xi}}}\,,
				\end{equation}
				где $r_0$ -- индекс опорного источника.
			\subsubsection{Преобразование Фурье}
				\par Преобразование Фурье~\cite{zorich} -- операция, сопоставляющая одной функции вещественной переменной другую функцию вещественной переменной. Эта новая функция описывает коэффициенты при разложении исходной функции на элементарные составляющие -- гармонические колебания с разными частотами. Преобразование Фурье функции $f$ вещественной переменной является интегральным и задаётся следующей формулой:
				\begin{equation}
					\hat{f}\pa{\xi}=\frac{1}{\sqrt{2\pi}}\int\limits_{-\infty}^\infty f\pa{t}e^{-it\xi}dt
				\end{equation}
				Также справедлива обратная формула, если интеграл имеет смысл:
				\begin{equation}
					f\pa{t}=\frac{1}{\sqrt{2\pi}}\int\limits_{-\infty}^\infty\hat{f}\pa{\xi}e^{it\xi}d\xi
				\end{equation}
				Важные свойства преобразования Фурье:
				\begin{subequations}
					\begin{enumerate}
						\item Линейность:
						\begin{equation}
							\widehat{\pa{\alpha f+\beta g}}=\alpha\hat{f}+\beta\hat{g},\quad\alpha,\beta\in\mathbb{R}
						\end{equation}
						\item Дифференцирование:
						\begin{equation}
							\widehat{f^{\pa{n}}}=\pa{i\xi}^n\hat{f}
						\end{equation}
					\end{enumerate}     
				\end{subequations}
		\subsection{Уровень звуковой экспозиции\label{sec::SEL}}
			\par Уровень звуковой экспозиции (sound exposure level, SEL) в области $\Omega=\left\{\pa{x,y,z}\bigr|x_0\leqslant x\leqslant x_1,y_0\leqslant y\leqslant y_1,z_0\leqslant z\leqslant z_b\right\}$ для отрезка частот $\left[f_1,f_2\right]$ некоторого источника задаётся следующим интегралом
			\begin{equation}\label{eq::SEL}
				\operatorname{SEL}\pa{x,y,z,f_1,f_2}=\int\limits_{f_1}^{f_2}\left|\hat{P}\pa{x,y,z,\xi\pa{f}}\right|^2df\,,
			\end{equation}
			где $\hat{P}$ -- функция сигнала в точке $\pa{x,y,z}$ \eqref{eq::receiver_signal}. Полученная величина является достаточно грубой, поэтому при численном вычислении  используется простой метод трапеций \cite{davis}
			\begin{equation}
				\operatorname{SEL}_{i,j,k}=d_f\sum\limits_{s=0}^{n_f-1}\left|\hat{P}_{i,j,k,s}\right|^2
			\end{equation}
			на равномерной сетке
			\begin{equation}
				\begin{gathered}
					\begin{aligned}
						\operatorname{SEL}_{i,j,k}&\approx\operatorname{SEL}\pa{x_i,y_j,z_k,f_1,f_2}\,,\\
						\hat{P}_{i,j,k,s}&=\hat{P}\pa{x_i,y_j,z_k,\xi\pa{f_s}}\,,\\
					\end{aligned}\\
					\begin{aligned}
						x_i&=x_0+id_x\,,\qquad i=\overline{0,n_x-1}\,,\\
						y_j&=y_0+jd_y\,,\qquad j=\overline{0,n_y-1}\,,\\
						z_i&=z_0+kd_z\,,\qquad k=\overline{0,n_z-1}\,,\\
						f_s&=f_1+sd_f\,,\qquad s=\overline{0,n_f-1}\,.
					\end{aligned}
				\end{gathered}
			\end{equation}
	\subsection{Выводы ко второй главе}
		\par В данной главе диссертации рассматривается теория широкоугольных модовых параболических уравнений в адиабатическом приближении и предлагается новый способ их решения. В основе этого метода лежит применение аппроксимации Паде к пропагатору уравнения горизонтальной рефракции. Для искусственного ограничения вычислительной области вводятся согласованные поглощающие слои или граничные условия прозрачности. Описывается лучевая теория для уравнения горизонтальной рефракции, на основе которой вводятся лучевые начальные условия, обладающие сколь угодно широкой апертурой. Описывается вычисление временных рядов в точках приёма при распространении импульсных акустических сигналов, а также уровней звуковой экспозиции. Предлагается способ расчёта колебательных ускорений, вносящий лишь незначительную добавку в общее время вычислений.
		\par Реализация описанных в данной главе методов, а также их валидация, описаны в третьей и четвёртой главах.
		\par Результаты второй главы опубликованы в работах \cite{jsv,jmse,acoustic_journal2023}.
\end{document}}