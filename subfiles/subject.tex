\documentclass[../document.tex]{subfiles}

\begin{document}
    \section{Описание предметной области и обоснование постановки задачи}
        \subsection{Оценка влияния антропогенных шумов}
            \par Моделирование трёхмерных звуковых полей применяется во многих областях исследования и освоения океана, требующих учёта множества параметров сложных неоднородных океанических волноводов.
            \par С развитием промышленности все больше расширяется хозяйственная деятельность человека, связанная с добычей нефти, газа и разнообразных биоресурсов в акватории мирового океана, в результате которой создаётся огромное количество антропогенных шумов, которые негативно сказываются на морской фауне \cite{noise1, noise2}. Таким образом, возникает задача оценивания и минимизации шумового загрязнения и его воздействия на мировой океан. Существуют два подхода к решению этой задачи: проведение измерений и моделирование. В первом случае проводится некоторое количество замеров плотности звука в воде, по которым строится интерполяционная картина звукового поля. Недостатком такого метода является дороговизна и сложность проведения измерений, которые также фактически могут быть получены только в точках среды на сетке с очень большими шагами по координатам, поэтому такие данные чаще всего используются для корректировки и проверки точности модельных данных. В свою очередь моделирование требует данных о положении и характеристиках источника звука, а также данных о свойствах среды: батиметрии, гидрологии и структуре дна. Основным преимуществом моделирования является возможность вычисления звукового поля как и уже существующих источников, так и планируемых, что позволяет заранее минимизировать влияние человека на океан. Недостатком такого метода является необходимость сбора и обработки изменяющихся данных о состоянии среды, что само по себе является сложной задачей.

            \par Мониторинг уровней акустической энергии, распределённой по большому морскому пространству, возникшей из антропогенных шумов, создаваемых в процессе различных индустриальных процессов на континентальном шельфе, является одной из областей, для которых применение трёхмерных моделей распространения звука является обязательным \cite{rutenko2007,nowacek2013,rutenko2012,bailey2014,Racca_Sakh2015,Racca_Crit_Mam2015,Broker_Thesis2021}. Действительно, не представляется возможным полностью покрыть интересуемое морское пространство приёмниками, поэтому несколько точечных опорных измерений должны быть использованы для восстановления звукового ландшафта морской среды. Также, зачастую требуется провести тщательное исследование звукового загрязнения морского пространства в реальном времени, чтобы в кратчайшие строки предоставить оценку его влияния на морскую фауну и проведеcnb подходящие меры смягчения последствий \cite{rutenko2007}. Такие требования накладывают существенные ограничения на производительность вычислительных программ для моделирования распространения звука. Например, в случае акустического мониторинга сейсмической разведки возникает задача моделирования широкополосного (10--250 Гц.) импульсного распространения в трёхмерной вычислительной области, представляющей собой морское пространство протягающееся на десятки километров в обоих горизонтальных направлениях \cite{rutenko2012,rutenko2019}. Многие существующие методы моделирования распространения звука не могут удовлетворить требованию того, что вычисления в данной области исследований должно проводиться почти в реальном времени. Например, известно, что использование современного высокоточного подхода, основанного на трёхмерных параболических уравнениях, требует 20 часов для вычисления акустического поля на частоте 25 Гц в эталонной задаче с подводным акустическим волноводом \cite{lin2013}. Таким образом, оказывается затруднительным полагаться на трёхмерные параболические уравнения для проведения моделирования широкополосных источников в разумные сроки.
        \subsection{Акустическая навигация}
            \par С каждым годом хозяйственная деятельность человека всё больше осуществляется с использованием автономных подводных аппаратов, требующих наличия стабильных систем навигации и связи, основанных на распространении звука, при этом привычные системы, основанные на электромагнитном излучении, неприменимы в подобных условиях \cite{navigation19,navigation20}. При разработке систем подводной акустической навигации возникает задача определения зон уверенного приёма и поиска взаимного расположения источников звукового сигнала таким образом, чтобы минимизировать зоны акустической тени. Также существует задача расчёта траекторий распространения звука на несущих частотах сигналов, с целью определения искривления по сравнению с геодезической на поверхности Земли для вычисления задержки звукового сигнала при осуществлении подводной навигации.
        \subsection{Шумы транспортных судов}
            \par В настоящее время эффективное и подходящее измерение шумов транспортных судов и оценка их влияния на окружающую среду является одной из главных проблем подводной акустики \cite{sh_ainslie2021,sh_jiang2020,sh_macgillivray2021,sh_wales2002,sh_merchant2012,sh_simard2016}. Широко известно, что судоходство является одним из основных антропогенных источников шума, влияющего на морские экосистемы, особенно в прибрежных районах близких к крупным морским портам. Систематическое подвержение высокому уровню звукового воздействия может негативно сказываться на популяции различных видов морских млекопитающих, рыб и даже безпозвоночных, поэтому существенные исследования фокусируется на определении приемлемого порога этого взаимодействия \cite{southall2017, southall2019marine, erbe2019}. С другой стороны шум, создаваемый транспортным судном, может быть источником информации о среде. Так, в \cite{gervaise2012,sh_simard2016} судоходный шум используется в качестве источника для возможности проведения геоакустического обращения параметров дна. В \cite{knobles2015} как спектр функции источника сухогруза, так и параметры морского дна были оценены с использованием статистического вычислительного метода, предполагающего, что судно может быть представлено в виде точечного источника, в то время как авторы \cite{tollefsen2021} одновременно оценивают спектр функции источника сухогруза и параметры модели морского дна, используя метод Байесовского обращения и представления торговых судов в виде нескольких точечных источников.

            \par В дополнение к прямым замерам, составление распределения уровня шума в больших морских пространствах требует подходящих инструментов моделирования распространения звука. В последнее время для данной задачи были разработаны несколько классов вычислительных методов, основанных на различных математических подходах, включающих лучевую теорию параболических уравнений \cite{lin2013,shtrum16}, метод потока энергии \cite{sertlek2014,sertlek2018} и модовые параболические уравнения  \cite{jsv,petrov2024} (то есть вертикальные моды, объединённые с двумерными параболическими уравнениями для вычисления амплитуд модового разложения поля). Большинство этих методов были подвержены тщательной верификации в различных эталонных задачах, включающих некоторые упрощённые модели среды (например, в \cite{jsv}) и монопольный (всенаправленный) точечный источник (во многих случаях тональный). Несмотря на впечатляющие вышеупомянутые успехи в разработке численных методов моделирования звукового поля в подводной акустике, достигнутые за последние 20 лет, до сих пор существует множество вопросов, возникающих из требований индустрии и реальных приложений, и относящихся к балансу эффективности и точности, а также корректному представлению различных источников шума в вычислительных моделях.
        \subsection{Обзор существующих методов решения}
            \par На данный момент существует несколько программных продуктов позволяющих получать численное решение уравнения Гельмгольца \eqref{eq::3DH}.\\ BELLHOP \cite{bellhop} и Traceo3D \cite{traceo}, основанные на методе суммирования Гауссовых пучков и лучевой теории распространения звука соответственно. Недостатком этих методов является использование геометроакустического приближения, которое является недостаточно точным при моделировании источников звука, имеющих частоту менее 1 кГц. Лин из океанографического института в Вудс-Хоуле и Стюрм из центральной школы Лиона в последнее десятилетие разработали закрытые комплексы программ, основанные на решении трёхмерного параболического уравнения \cite{isakson14,lin12,shtrum16}, однако решение таких уравнений требует запредельных затрат памяти и времени, вычисление решения даже самых простых задач занимает не менее суток.
        \subsection{Требования к модели и её программной реализации}
            \par Существующие на данный момент программные комплексы в основном сфокусированы на решении какой-то одной мелкой задачи, зачастую являющейся часть чего-то большего, или же направлены на решение какой-то одной конкретной задачи. Также, многие из них не позволяют выполнять вычисления за разумное время. Поэтому возникает необходимость в разработке новой программы, позволяющей решать некоторый спектр задач, не имея привязанности к определённым параметрам. Основными требованиями к разрабатываемой программе являются:
            \begin{enumerate}
                \item Реализация численных схем решения модовых параболических уравнений.
                \item Моделирование звукового поля на трёхмерной сетке.
                \item Возможность использования как готовых коэффициентов уравнения, так
                и коэффициентов, вычисленных с помощью пакета Cambala \cite{cambala}, с указанием необхо­димых параметров: плотности среды, батиметрии, гидрологии и др.
                \item Проведение трассировки горизонтальных лучей, соответствующих вертикальным модам.
                \item Вычисление временного ряда импульса звукового сигнала в произвольных точках среды.
                \item Оценка уровня шума с использованием интегральной характеристики звуковой экспозиции.
                \item Высокая скорость работы по сравнению с альтернативными методами моделирования.
            \end{enumerate}
\end{document}
