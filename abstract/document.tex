\documentclass[defernumbers,nosortbib]{fefu}

\usepackage{bm}
\usepackage{minted}
\usepackage{amssymb}
\usepackage{subfiles}
\usepackage{nicefrac}

\setmainfont{CMU Serif}
\addbibresource{../references.bib}
\graphicspath{{../images/}}

\fefuloadstyle[../]{poi_phd}

\newcounter{sectionscount}
\newcounter{allpagescount}
\newcounter{textpagescount}
\newcounter{figurescount}
\newcounter{tablescount}
\newcounter{citescount}
\newcounter{bibpagescount}
\newcommand{\handlecounts}{%
    \input{../counters.tex}%

	\setcounter{sectionscount}{\pgfkeysvalueof{/counters/sectionscount}}%
    \setcounter{allpagescount}{\pgfkeysvalueof{/counters/allpagescount}}%
    \setcounter{textpagescount}{\pgfkeysvalueof{/counters/textpagescount}}%
    \setcounter{figurescount}{\pgfkeysvalueof{/counters/figurescount}}%
    \setcounter{tablescount}{\pgfkeysvalueof{/counters/tablescount}}%
    \setcounter{citescount}{\pgfkeysvalueof{/counters/citescount}}%
    \setcounter{bibpagescount}{\pgfkeysvalueof{/counters/bibpagescount}}%
}

\newcommand{\pa}[1]{\left(#1\right)}
\newcommand{\code}[2][text]{\mintinline{#1}{#2}}
\newcommand{\niceref}[2]{\hyperref[#1]{#2 \ref{#1}}}
\newcommand{\dd}{\mathrm{d}}
\newcommand{\introductiontitle}{Общая характеристика работы}

\makeatletter
\@FEFUnotoctrue
\titleformat{name=\section,numberless}[block]
{\bfseries\large}{}{0pt}{#1}
\makeatother

\defbibfilter{mine}{keyword=myarticles or keyword=myproceedings}

\begin{document}
	\subfile{../subfiles/introduction.tex}
	\newpage
	\section*{Содержание работы}
	\par\textbf{Во Введении} обоснована актуальность диссертационной работы, сформулирована цель и аргументирована научная новизна исследований, показана
	практическая значимость полученных результатов, представлены выносимые
	на защиту научные положения.
	\par \textbf{В первой главе} приводится краткое описание задач, для которых моделирование трёхмерных звуковых полей
	НИЧЕГО
	\par \textbf{Вторая глава} посвящена описанию математической части поставленной задачи и методам её решения. Приводится детальное описание предлагаемого алгоритма моделирования. Звуковое поле ищется в форме модового разложения
	\begin{equation}
		p\pa{x,y,z}=\sum\limits_{j=1}^\infty A_j\pa{x,y}\varphi_j\pa{z}\,,
	\end{equation}
	где $A_j\pa{x,y}$ называются модовыми амплитудами и удовлетворяют уравнению горизонтальной рефракции
	\begin{equation}
		\frac{\partial^2A_j\pa{x,y}}{\partial x^2}+\frac{\partial^2 A_j\pa{x,y}}{\partial y^2}+k_j^2\pa{x,y}A_j\pa{x,y}=-\delta\pa{x}\delta\pa{y}\varphi_j\pa{z_s,0,0}\,,
	\end{equation}
	при этом модовые функции $\varphi_j\pa{z,x,y}$ и соответствующие им волновые числа $k_j\pa{x,y}$ являются решениями соответствующей спектральной задачи при фиксированных $x$ и $y$. Решение уравнения горизонтальной рефракции далее сводится к решению псевдодифференциального модового параболического уравнения вида
	\begin{gather}\label{eq::PDMPE}
		A_j\pa{x,y}=e^{ik_{j,0}x}\mathcal{A}_j\pa{x,y}\,,\\
		\frac{\partial\mathcal{A}_j\pa{x,y}}{\partial x}=ik_{j,0}\pa{\sqrt{1+L_j}-1}\mathcal{A}_j\pa{x,y}\,,\\
		k_{j,0}^2L_j=\frac{\partial^2}{\partial y^2}+k_j^2\pa{x,y}-k_{j,0}^,.
	\end{gather}
	\par В \textbf{разделе 2.3} подробно рассматриваются аппроксимация псевдодифференциального оператора и дискретизация уравнения \eqref{eq::PDMPE} по координате $x$ с использованием аппроксимации Паде и метода Split-step Pad\'e. Последний заключается в формальном решении уравнения в виде на равномерной сетке $\Delta x=h$
	\begin{equation}
		\mathcal{A}_j^{n+1}=e^{ik_{j,0}h\pa{\sqrt{1+L_j}}}\mathcal{A}_j^n\,.
	\end{equation}
	Затем аппроксимации Паде применяется к операторной экспоненте
	\begin{equation}
		e^{ik_{j,0}h\pa{\sqrt{1+L_j}}}\approx\frac{U_{l,m}\pa{L_j}}{W_{l,m}\pa{L_j}}=a_{l,m}^0+\sum\limits_{i=1}^{p=\max\left\{l,m\right\}}\frac{a_{l,m}^i}{1+b_{l,m}^iL_j}\,.
	\end{equation}
	Далее проводится дискретизация оператора $L_j$ на равномерной сетке $\Delta y=\delta$ и приводится численная схема решения уравнения \eqref{eq::PDMPE}
	\begin{gather}
		\mathcal{A}_j^{n+1,q}=a_{l,m}^0\mathcal{A}_j^{n,q}+\sum\limits_{i=1}^{p}a_{l,m}^i\mathcal{B}_{j,i}^{n+1,q}\,,\\
		\pa{1+b_{l,m}^iL_j^\delta}\mathcal{B}_{j,i}^{n+1,q}=\mathcal{A}_j^{n,q},\quad i=\overline{1,p}\,,\\
		\underset{\alpha_{j,i}}{\underbrace{\frac{b_{l,m}^i}{k_{j,0}^2\delta^2}}}\mathcal{B}_{j,i}^{n+1,q-1}+\underset{\beta_{j,i}}{\underbrace{\pa{1+\frac{b_{l,m}^i}{k_{j,0}^2}\pa{k_j^2-k_{j,0}^2-\frac{2}{\delta^2}}}}}\mathcal{B}_{j,i}^{n+1,q}+\underset{\gamma_{j,i}}{\underbrace{\frac{b_{l,m}^i}{k_{j,0}^2\delta^2}}}\mathcal{B}_{j,i}^{n+1,q+1}=\mathcal{A}_j^{n,q}\,.
	\end{gather}
    Таким образом решение уравнения может быть получено путём последовательного обращения трёхдиагональных матриц.
	\par В \textbf{разделе 2.5} рассматриваются граничные условия, используемые для искусственного ограничения вычислительной области. Первыми рассматриваются согласованные поглощающие слои, заключающиеся в расширении вычислительной области на $\varepsilon$ с обоих сторон по координате $y$ и заменой оператора $L_j$ на оператор $L_j^{PML}$
	\begin{equation}
		k_{j,0}^2L_j^{PML}=\frac{1}{1+i\beta\pa{y}}\frac{\partial}{\partial y}\frac{1}{1+i\beta\pa{y}}\frac{\partial}{\partial y}+k_j^2+k_{j,0}^2\,,
	\end{equation}
	где $\beta\pa{y}$ -- некоторая гладкая функция, монотонно убывающая на интервале $\left[y_0-\varepsilon,y_0\right)$, возрастающая на $\left(y_1,y_1+\varepsilon\right]$, и $\beta\pa{y}=0$ при $y\in\left[y_0,y_1\right]$. Затем рассматриваются граничные условия прозрачности имеющие следующий вид на правой границе области $y_Q=y_1$
	\begin{equation}
		\bm{\Psi}_j^{n+1,Q}-\bm{\Psi}_j^{n+1,Q-1}=\sum\limits_{l=1}^{n+1}\mathbf{D}_j^{n+1-l}\bm{\Psi}_j^{l,Q}\,,
	\end{equation}
	где $\bm{\Psi}_j^{n,q}=\pa{\mathcal{A}_j^{n,q},\mathcal{B}_{j,1}^{n,q},\dots,\mathcal{B}_{j,1}^{n,q}}^T\in\mathbb{C}^{p+1}$. Вычисление коэффициентов $\textbf{D}_j$ подробно описано в тексте работы. Граничные условия на левой границе имеют аналогичную форму.
	\par В \textbf{разделе 2.6} описывается лучевая теория распространения звука и численный метод трассировки горизонтальных лучей, соответствующих вертикальным модам. Далее в \textbf{разделе 2.7} вводятся лучевые начальные условия, лучше удовлетворяющие широкоугольным свойствам аппроксимации Паде по сравнению с начальными условиями Гаусса и Грина. Условия ставятся на некотором небольшом расстоянии $x_0$ от источника в предположении $k_j=k_j\pa{y}$, и имеют следующий вид
	\begin{gather}
        \mathcal{A}_j\pa{x_0,y}=M_j\pa{x_0,y}e^{ik_{j,0}S_j\pa{x_0,y}}\,,\\
		S_j\pa{y\pa{l,\alpha}}=\int\limits_0^ln_j\pa{y\pa{\ell,\alpha}}d\ell\,,\\
        M_j\pa{y\pa{l,\alpha}}=\frac{M_{j,0}}{n_j\pa{y\pa{l,\alpha}}}\sqrt{\frac{\cos\alpha}{\nicefrac{\partial y\pa{l,\alpha}}{\partial\alpha}}}\,,\\
        M_{j,0}=\nicefrac{e^{\nicefrac{i\pi}{4}}}{\sqrt{8\pi k_{j,0}}}\,,\quad n_j\pa{y}=\nicefrac{k_j\pa{y}}{k_{j,0}}\,.\nonumber
	\end{gather}
    Так как расстояние $x_0$ является небольшим (несколько десятков метров) имеет место предложить однородность $k_j$ по обеим координатам $k_j\pa{x,y}\equiv k_{j,0}$, тогда
    \begin{equation}
        \begin{gathered}
            S_j\pa{y}=r\pa{y}\,,\quad M_j\pa{y}=\nicefrac{M_{j,0}}{r\pa{y}}\,,\\
            r\pa{y}=\sqrt{x_0^2+y^2}\,.
        \end{gathered}
    \end{equation}
	\newpage
    \begin{refsegment}
        \printbibliography[title=Цитированная литература,notkeyword={myarticles},notkeyword={myproceedings}]
    \end{refsegment}
    \begin{refsegment}
        \newrefcontext[labelprefix=A]
        \printbibliography[title=Список публикаций,filter=mine,resetnumbers=true]
    \end{refsegment}
\end{document}
