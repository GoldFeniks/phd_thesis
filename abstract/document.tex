\documentclass[defernumbers,nosortbib]{fefu}

\usepackage{bm}
\usepackage{minted}
\usepackage{amssymb}
\usepackage{subfiles}
\usepackage{nicefrac}
\usepackage{subcaption}
\usepackage{placeins}

\setmainfont{CMU Serif}
\addbibresource{../references.bib}
\graphicspath{{../images/}}

\fefuloadstyle[../]{poi_phd}

\newcounter{sectionscount}
\newcounter{allpagescount}
\newcounter{textpagescount}
\newcounter{figurescount}
\newcounter{tablescount}
\newcounter{citescount}
\newcounter{bibpagescount}
\newcommand{\handlecounts}{%
    \input{../counters.tex}%

	\setcounter{sectionscount}{\pgfkeysvalueof{/counters/sectionscount}}%
    \setcounter{allpagescount}{\pgfkeysvalueof{/counters/allpagescount}}%
    \setcounter{textpagescount}{\pgfkeysvalueof{/counters/textpagescount}}%
    \setcounter{figurescount}{\pgfkeysvalueof{/counters/figurescount}}%
    \setcounter{tablescount}{\pgfkeysvalueof{/counters/tablescount}}%
    \setcounter{citescount}{\pgfkeysvalueof{/counters/citescount}}%
    \setcounter{bibpagescount}{\pgfkeysvalueof{/counters/bibpagescount}}%
}

\newcommand{\pa}[1]{\left(#1\right)}
\newcommand{\code}[2][text]{\mintinline{#1}{#2}}
\newcommand{\niceref}[2]{\hyperref[#1]{#2 \ref{#1}}}
\newcommand{\dd}{\mathrm{d}}
\newcommand{\introductiontitle}{Общая характеристика работы}

\makeatletter
\@FEFUnotoctrue
\titleformat{name=\section,numberless}[block]
{\bfseries\large}{}{0pt}{#1}
\makeatother

\defbibfilter{mine}{keyword=myarticles or keyword=myproceedings or keyword=myprograms}
\disablehyphenation
\usepackage{tabularx}
\begin{document}
	\subfile{../subfiles/introduction.tex}
	\newpage
	\section*{Содержание работы}
	\par\textbf{Во Введении} обоснована актуальность диссертационной работы, сформулирована цель и аргументирована научная новизна исследований, показана
	практическая значимость полученных результатов, представлены выносимые
	на защиту научные положения.
	\par \textbf{В первой главе} приводится краткий обзор некоторых классов задач акустики океана, для решения которых моделирование трёхмерных звуковых полей играет важную роль, в частности рассматриваются задачи акустического мониторинга и дальней акустической навигации. Кратко обсуждаются известные средства моделирования трёхмерных акустических полей и формируются требования к модели, которая разрабатывается в настоящей диссертации.   
	\par \textbf{Вторая глава} посвящена описанию математической части поставленной задачи и методам её решения. В \textbf{разделе 2.1} описывается математическая постановка задачи. Звуковое поле ищется в форме модового разложения
	\begin{equation}
		p\pa{x,y,z}=\sum\limits_{j=1}^\infty A_j\pa{x,y}\varphi_j\pa{z}\,,
	\end{equation}
	где $A_j\pa{x,y}$ называются модовыми амплитудами и в адиабатическом приближении удовлетворяют уравнению горизонтальной рефракции
	\begin{equation}
		\frac{\partial^2A_j\pa{x,y}}{\partial x^2}+\frac{\partial^2 A_j\pa{x,y}}{\partial y^2}+k_j^2\pa{x,y}A_j\pa{x,y}=-\delta\pa{x}\delta\pa{y}\varphi_j\pa{z_s,0,0}\,,
	\end{equation}
	при этом модовые функции $\varphi_j\pa{z,x,y}$ и соответствующие им волновые числа $k_j\pa{x,y}$ являются решениями соответствующей спектральной задачи при фиксированных $x$ и $y$. Далее в \textbf{разделе 2.2} решение уравнения горизонтальной рефракции сводится к решению псевдодифференциального модового параболического уравнения вида (в приближении однонаправленного распространения)
	\begin{gather}\label{eq::PDMPE}
		A_j\pa{x,y}=e^{ik_{j,0}x}\mathcal{A}_j\pa{x,y}\,,\\
		\frac{\partial\mathcal{A}_j\pa{x,y}}{\partial x}=ik_{j,0}\pa{\sqrt{1+L_j}-1}\mathcal{A}_j\pa{x,y}\,,\\
		k_{j,0}^2L_j=\frac{\partial^2}{\partial y^2}+k_j^2\pa{x,y}-k_{j,0}^,.
	\end{gather}
	\par В \textbf{разделе 2.3} подробно рассматриваются аппроксимация псевдодифференциального оператора и дискретизация уравнения \eqref{eq::PDMPE} по координате $x$ с использованием аппроксимации Паде и метода Split-step Pad\'e (SSP). Последний заключается в формальном решении уравнения на равномерной сетке $\Delta x=h$ в виде 
	\begin{equation}
		\mathcal{A}_j^{n+1}=e^{ik_{j,0}h\pa{\sqrt{1+L_j}}}\mathcal{A}_j^n\,.
	\end{equation}
	Затем аппроксимации Паде применяется к операторной экспоненте
	\begin{equation}
		e^{ik_{j,0}h\pa{\sqrt{1+L_j}}}\approx\frac{U_{l,m}\pa{L_j}}{W_{l,m}\pa{L_j}}=a_{l,m}^0+\sum\limits_{i=1}^{p=\max\left\{l,m\right\}}\frac{a_{l,m}^i}{1+b_{l,m}^iL_j}\,.
	\end{equation}
	Далее в \textbf{разделе 2.4} проводится дискретизация оператора $L_j$ на равномерной сетке $\Delta y=\delta$ и приводится численная схема решения уравнения \eqref{eq::PDMPE}
	\begin{gather}\label{eq::numerical_scheme}
		\mathcal{A}_j^{n+1,q}=a_{l,m}^0\mathcal{A}_j^{n,q}+\sum\limits_{i=1}^{p}a_{l,m}^i\mathcal{B}_{j,i}^{n+1,q}\,,\\
		\pa{1+b_{l,m}^iL_j^\delta}\mathcal{B}_{j,i}^{n+1,q}=\mathcal{A}_j^{n,q},\quad i=\overline{1,p}\,,\\
		\underset{\alpha_{j,i}}{\underbrace{\frac{b_{l,m}^i}{k_{j,0}^2\delta^2}}}\mathcal{B}_{j,i}^{n+1,q-1}+\underset{\beta_{j,i}}{\underbrace{\pa{1+\frac{b_{l,m}^i}{k_{j,0}^2}\pa{k_j^2-k_{j,0}^2-\frac{2}{\delta^2}}}}}\mathcal{B}_{j,i}^{n+1,q}+\underset{\gamma_{j,i}}{\underbrace{\frac{b_{l,m}^i}{k_{j,0}^2\delta^2}}}\mathcal{B}_{j,i}^{n+1,q+1}=\mathcal{A}_j^{n,q}\,.
	\end{gather}
    Таким образом решение уравнения может быть получено путём последовательного обращения трёхдиагональных матриц.
	\par В \textbf{разделе 2.5} рассматриваются граничные условия, используемые для искусственного ограничения вычислительной области. Первыми рассматривается метод согласованных поглощающих слоёв, заключающийся в расширении вычислительной области на $\varepsilon$ с обеих сторон по координате $y$ и замене оператора $L_j$ на оператор $L_j^{PML}$
	\begin{equation}
		k_{j,0}^2L_j^{PML}=\frac{1}{1+i\beta\pa{y}}\frac{\partial}{\partial y}\frac{1}{1+i\beta\pa{y}}\frac{\partial}{\partial y}+k_j^2+k_{j,0}^2\,,
	\end{equation}
	где $\beta\pa{y}$ -- некоторая гладкая функция, монотонно убывающая на интервале $\left[y_0-\varepsilon,y_0\right)$, возрастающая на $\left(y_1,y_1+\varepsilon\right]$, и $\beta\pa{y}=0$ при $y\in\left[y_0,y_1\right]$. Затем выведены граничные условия прозрачности имеющие следующий вид на правой границе области $q=Q$
	\begin{equation}\label{eq::tbc}
		\bm{\Psi}_j^{n+1,Q}-\bm{\Psi}_j^{n+1,Q-1}=\sum\limits_{l=1}^{n+1}\mathbf{D}_j^{n+1-l}\bm{\Psi}_j^{l,Q}\,,
	\end{equation}
	где $\bm{\Psi}_j^{n,q}=\pa{\mathcal{A}_j^{n,q},\mathcal{B}_{j,1}^{n,q},\dots,\mathcal{B}_{j,1}^{n,q}}^T\in\mathbb{C}^{p+1}$. Вычисление коэффициентов $\textbf{D}_j$ подробно описано в тексте работы. Граничные условия на левой границе имеют аналогичную форму. Заметим, что условия \eqref{eq::tbc} для \eqref{eq::numerical_scheme} обеспечивают идеальную прозрачную границу уравнения, то есть согласуются с численной схемой.
	\par В \textbf{разделе 2.6} кратко описывается лучевая теория распространения звука и численный метод трассировки горизонтальных лучей, соответствующих вертикальным модам. Далее в \textbf{разделе 2.7} вводятся лучевые начальные условия, лучше удовлетворяющие широкоугольным свойствам аппроксимации Паде по сравнению с начальными условиями Гаусса и Грина. Условия ставятся на некотором небольшом расстоянии $x_0$ от источника в предположении $k_j=k_j\pa{y}$, и имеют следующий вид
	\begin{gather}
        \mathcal{A}_j\pa{x_0,y}=M_j\pa{x_0,y}e^{ik_{j,0}S_j\pa{x_0,y}}\,,\\
		S_j\pa{y\pa{l,\alpha}}=\int\limits_0^ln_j\pa{y\pa{\ell,\alpha}}d\ell\,,\\
        M_j\pa{y\pa{l,\alpha}}=\frac{M_{j,0}}{n_j\pa{y\pa{l,\alpha}}}\sqrt{\frac{\cos\alpha}{\nicefrac{\partial y\pa{l,\alpha}}{\partial\alpha}}}\,,\\
        M_{j,0}=\nicefrac{e^{\nicefrac{i\pi}{4}}}{\sqrt{8\pi k_{j,0}}}\,,\quad n_j\pa{y}=\nicefrac{k_j\pa{y}}{k_{j,0}}\,.\nonumber
	\end{gather}
    Так как расстояние $x_0$ является небольшим (несколько десятков метров) имеет место предложить однородность $k_j$ по обеим координатам $k_j\pa{x,y}\equiv k_{j,0}$, тогда
    \begin{equation}
        \begin{gathered}
            S_j\pa{y}=r\pa{y}\,,\quad M_j\pa{y}=\nicefrac{M_{j,0}}{r\pa{y}}\,,\\
            r\pa{y}=\sqrt{x_0^2+y^2}\,.
        \end{gathered}
    \end{equation}
    \par В \textbf{разделе 2.8} рассматривается расчёт временных рядов в точках приёма при распространении импульсных акустических сигналов. Импульсный сигнал получается из построения спектра сигнала в приёмниках с использованием звукового давления для каждой рассматриваемой частоты и последующим применением обратного преобразования Фурье. При этом функция или спектр источника могут быть известны напрямую или рассчитаны через импульс или спектр в некоторой опорной точке волновода. В \textbf{разделе 2.9} описывается вычисление уровня звукового воздействия (SEL), представляющего собой интеграл спектра сигнала на диапазоне частот $\left[f_1,f_2\right]$. В \textbf{разделе 2.10} приводится описание методики расчёта колебательных скоростей и ускорения, выражаемых через градиент звукового давления по пространственным координатам, в рамках теории модовых параболических уравнений. Ввиду использования модового разложения имеет место переход к градиенту медленно изменяющихся функций, градиент которых может быть численно получен с использованием простых конечных разностей.
    \par Результаты второй главы опубликованы в работах \cite{jsv,jmse,acoustic_journal2023}.
    \par В \textbf{третьей главе} приводится описание программной реализации \cite{ample} описанных методов моделирования распространения звука, а также их валидации и тестирования на модельных задачах. Реализация комплекса программ была выполнена на языке C++20 и доступна в открытом доступе по адресу: \url{https://github.com/GoldFeniks/Ample}. \textbf{Раздел 3.1} посвящён детальному описанию структуры проекта, а также интерфейсу взаимодействия с программой. Программа \code{AMPLE} позволяет вычислять поле звукового давления, модовые функции и волновые числа, траектории распространения горизонтальных лучей, уровни поля звукового воздействия, временные ряды в точках приёма для данного сигнала, излучаемого источником, поле колебательных ускорений. Основным входным файлом программы является конфигурационный файл в формате JSON, содержащий информацию о параметрах среды (батиметрия, гидрология, плотность и т.п.), сетки вычислительной области, координатах приёмника и источника, виде начальных условий и параметрах численной схемы решения. Выходные данные записываются в папку, указанную аргументами командной строки. В результате работы программы будут созданы файлы \code{config.json} и \code{meta.json}, а также папки, содержащие многомерные выходные данные. Файл \code{config.json} содержит информацию из конфигурационного файла, указанного при запуске программы, а также значения параметров, которые были использованы во время работы, но не были явно указаны. Файл \code{meta.json} содержит информацию о работе программы, такие как аргументы командной строки, время выполнения, пути к сохранённым результатам работы. Так, например, результаты моделирования трёхмерного поля акустического давления будут сохранены в бинарные или текстовые файлы \code{solution/frequency.(txt|bin)}, где \code{frequency} -- частота при которой были получены данные. Дополнительно файл \code{solution/meta.json} содержит описание выведенных акустических полей -- их перечисление и размеры.
    \par В \textbf{разделе 3.2} рассматривается тестирование и валидация разработанного комплекса программ. В качестве одной из задач было рассмотрено моделирование распространения звука в стандартном клиновидном волноводе мелкого моря, часто использующийся для валидации методов моделирования \cite{jensen} (см. \niceref{fig::wedge}{Рисунок}). Звуковое поле было вычислено с порядком аппроксимации $p=13$, результаты вычислений отображены на \niceref{fig::wedge_field}{Рисунке}. Было проведено сравнение полученного решения (SSP) с решением, использующим широкоугольную аппроксимацию Клаербоута (WAMPE) \cite{bachelor}, а также с решением методом изображений \cite{deane,tang}. Как показано на \niceref{fig::wedge_compare}{Рисунке}, решения всех методов почти совпадают, не смотря на адиабатическую природу модовых параболических уравнений, при этом большая апертура SSP метода ещё сильнее приближает решение к решению методом изображений вдали от источника.
    \par Далее было проведено сравнение горизонтальных лучей, вычисленных с использованием программы AMPLE \cite{ample}, а также лучей, полученных из закона Снеллиуса в виде криволинейного интеграла второго рода и выпущенных под углом $\psi$
    \begin{equation}\label{eq::snell_ray}
    	x\pa{S}=\int\limits_0^S\frac{y^\prime_t\pa{s}\dd s}{\sqrt{\frac{k_j^2\pa{y\pa{s}}}{k_j^2\pa{0}\cos^2\pa{\psi}}-1}}\,.
    \end{equation}
    Результаты сравнения, показавшие идеальную сходимость, показаны на \niceref{fig::rays}{Рисунке}.
    \par Результаты третьей главы опубликованы в работах и представлены на конференциях \cite{acoustic_journal2021,acoustic_journal2023,dd}
    \begin{figure}[h]
    	\begin{minipage}[h][0.33\textheight]{0.4\textwidth}
    		\centering
    		\vfill
    		\includegraphics[width=0.95\textwidth]{wedge.png}
    		\vfill
    		\captionof{figure}{Схематическое изображение подводного клина\label{fig::wedge}}
    	\end{minipage}
    	\hfill
    	\begin{minipage}[h][0.33\textheight]{0.5\textwidth}
    		\centering
    		\includegraphics[width=\textwidth]{wedge_n13.pdf}
    		\vfill
    		\captionof{figure}{Акустическое поле (в дБ отн. 1 м.) в клиновидном прибрежном волноводе мелкого моря $z=30\ \text{м.}$\label{fig::wedge_field}}
    	\end{minipage}
    \end{figure}
    \begin{figure}[h]
    	\centering
    	\begin{subfigure}[t]{0.88\textwidth}
    		\centering
    		\includegraphics[width=0.9\textwidth]{wedge_comp.pdf}
    	\end{subfigure}
    	\hfill
    	\begin{subfigure}[t]{0.88\textwidth}
    		\centering
    		\includegraphics[width=0.9\textwidth]{wedge_comp_close.pdf}
    	\end{subfigure}
    	\caption{Сравнение результатов вычисления акустического поля (в дБ отн. 1 м.) в мелком море с подводным клином при $y=0\ \text{км}\,, z=30\ \text{м}$.\label{fig::wedge_compare}}
    \end{figure}
    \begin{figure}[h]
    	\centering
    	\includegraphics[width=\textwidth]{rays1.pdf}
    	\caption{Сравнение трассировки лучей с использованием программы AMPLE (сплошная кривая) и численного интегрирования \eqref{eq::snell_ray} (пунктирная кривая) для первой распространяющейся моды. \label{fig::rays}}
    \end{figure}
    \FloatBarrier
    \par \textbf{Глава 4} посвящена 
	\newpage
    \begin{refsegment}
        \printbibliography[title=Цитированная литература,notkeyword={myarticles},notkeyword={myproceedings},notkeyword={myprograms}]
    \end{refsegment}
    \begin{refsegment}
        \newrefcontext[labelprefix=A]
        \printbibliography[title=Список публикаций,filter=mine,resetnumbers=true]
    \end{refsegment}
\end{document}
