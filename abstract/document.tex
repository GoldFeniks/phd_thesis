\documentclass{fefu}

\begin{document}
	\section*{Общая характеристика работы}
	\paragraph*{Актуальность темы исследования}
	\par В настоящее время активно развиваются методы моделирования распространения звука с учётом трёхмерных эффектов.
	\par Такие эффекты как отражение, преломление, диффракция и рассеяние акустической энергии в трёхмерном пространстве, то есть одновременно в вертикальной и горизонтальной плоскостях, были представлены в литературе на протяжении  нескольких десятилетий. Описание наиболее ранних работ представлено, например в книге \cite{lee1995} и обзорной статье \cite{tolstoy1996}. В дальнейшем были разработаны первые эффективные и результативные модели, например \cite{smith1999,sturm2003,heaney2016}. Необходимость учитывать трёхмернеы эффекты также была показана экспериментально 
	\paragraph*{Степень разработанности темы исследования}
	\paragraph*{Цели и задачи диссертационной работы}
	\paragraph*{Научная новизна}
	\paragraph*{Теоретическая и практическая значимость}
	\paragraph*{Методология и методы исследования}
	\paragraph*{Положения, выносимые на защиту}
	\begin{enumerate}
		\item Разработана и апробирована методика численного решения псевдодифференциальных модовых параболических уравнений с искусственным ограничением расчётной области путём постановки граничных условий прозрачности или добавления к ней согласованных поглощающих слоёв.  
		\item Разработан комплекс программ на языке программирования C++, который может быть использован для моделирования распространения тональных и импульсных сигналов, а также вычисления скалярных и векторных акустических полей антропогенных шумов в океане с возможностью учёта батиметрических и гидрологических данных и структуры слоёв дна, и ориентированный на максимальную производительность.
		\item Моделирование антропогенных шумов, связанных с сейсморазведочными работами и судоходством, проведённое с использованием разработанного комплекса прикладных программ, позволило добиться согласия уровней акустической экспозиции (SEL) с данными прямых измерений с точностью до 1 дБ. и согласия значений распределения энергии в децидекадных частотных полосах на различных расстояниях от источника шума с точностью до 5 дБ. для диапазона частот, в котором сосредоточена большая часть энергии сигнала. 
		\item Всё тлен
	\end{enumerate}
	\paragraph*{Степень достоверности и апробация результатов}
	\par Методы, описанные в работе, а также их программная реализация, были протестированы на множестве модельных задач и на экспериментах с использованием данных подводных акустических зондов. Достоверность результатов обусловливается хорошей их согласованностью с известными методами моделирования и результатами натурных измерений. Основные результаты диссертации докладывались на следующих конференциях: на конференции <<Days on Diffraction>> (Санкт-Петербург, 2019, дистанционно 2022), на сессиях Российского акустического общества (дистанционно 2022, 2023), на конференции <<Океанологические исследования>> (Владивосток, 2023). 
	\paragraph*{Публикации}
	\par Материалы диссертации опубликованы в 11 печатных работах, из них 5 в в рецензируемых научных журналах, включённых в перечень ВАК, а также индексируемых в международных библиографических базах данных Scopus ("Скопус") и Web of Science ("Сеть науки").
	\paragraph*{Личный вклад автора}
	\par Автор работы принмал участие в разработке численных методов решения модовых параболических уравнений, в том числе в областях с открытыми границами. Автором работы была проведена программная реализация численных методов решения уравнения Гельмгольца с использованием широкоугольных параболических уравнений. Также он принимал участие в разработке, реализации и апробации методики учёта данных опорных измерений при моделировании натурных акустических экспериментов.
	\paragraph*{Структура и объём диссертации}
	\paragraph*{Благодарности}
\end{document}